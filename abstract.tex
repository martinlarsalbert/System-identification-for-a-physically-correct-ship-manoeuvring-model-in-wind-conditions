% Move 1 - Background/introduction/situation
System identification offers ways to obtain proper models describing a ship's dynamics in real operational conditions but poses significant challenges, such as the multicollinearity and generality of the identified model. 

% Move 2 - Present research/purpose
This paper proposes a new physics-informed ship manoeuvring model, where a deterministic semi-empirical rudder model has been added, to guide the identification toward a physically correct hydrodynamic model.  
This is an essential building block to distinguish the hydrodynamic modeling uncertainties from wind, waves, and currents---in real sea conditions---which is particularly important for ships with wind-assisted propulsion.
In the physics-informed manoeuvring modeling framework, a systematical procedure is developed to establish various force/motion components within the manoeuvring system by inverse dynamics regression. 

% Move 3 - Methods/materials/subjects/procedures
The novel test case wind-powered pure car carrier (wPCC) assesses the physical correctness. First, a reference model, assumed to resemble the physically correct kinetics, is established via parameter identification on virtual captive tests. Then, the model tests are used to build both the physics-informed model and a physics-uninformed mathematical model for comparison.

% Move 4 - Results/findings
All models predicted the zigzag tests with satisfactory agreement. Thus, they can indeed be considered as being mathematically correct. However, introducing a semi-empirical rudder model seems to have guided the identification toward a more physically correct calm water hydrodynamic model, having lower multicollinearity and better generalization.

% Move 5 - Discussion/conclusion/significance