% Move 1 - Background/introduction/situation
Modern ships gather vast amounts of kinematic data, including high-precision GPS, accelerometers, and inclinometers, offering opportunities to optimize ship operations. Predictive modeling using machine learning (ML) techniques is commonly employed for systems with unknown physics, making ML an adaptable choice for forecasting. However, in cases where prior knowledge of a system's physics and structure is available, such as ship manoeuvring, system identification emerges as a valuable alternative. By leveraging established knowledge, system identification models establish causal relationships between variables, enabling optimization of alternative ship maneuvers.

% Move 2 - Present research/purpose
The correctness of identified models can be mathematical, yielding favorable results on training and test datasets, yet lacking physical accuracy. This discrepancy becomes evident when a model trained in calm water conditions is exposed to wind during maneuvers. This paper explores how models with varying levels of physical correctness respond to such scenarios.

% Move 3 - Methods/materials/subjects/procedures
To evaluate the physical correctness of identified models, three unique datasets from a wind-powered pure car carrier (WPCC) are employed. Data from virtual captive tests (VCT), calculated via computational fluid dynamics (CFD) at various steady-state drift angles, yaw rates, and rudder angles, establish the physically correct kinetics. System identification is conducted through inverse dynamics regression (IDR) on a second dataset containing a series of manoeuvring model tests with a free model. The impact of wind forces is studied on a third dataset - where wind forces are included.

% Move 4 - Results/findings
An innovative addition to this study is the incorporation of a semi-empirical lifting line rudder model to guide IDR in finding a more physically accurate model. This model exhibits superior performance in wind tests compared to models with less physically accurate rudder components.

% Move 5 - Discussion/conclusion/significance
A drift angle is needed when the ship is traveling on a straight course in wind - to counteract the wind forces. This wind state is very rare in calm water manoeuvring tests, where the drift angle is almost exclusively accompanied by a yaw rate. A pure mathematical model, which has been fitted to manoeuvring tests, will have a poor representation of the wind state due to multicollinearity. Multicollinearity is a well-known challenge in the regression of parameterized manoeuvring models, which can result in mathematically correct yet physically incorrect models. Guiding models toward increased physical accuracy -- by adding a semi-empirical rudder model -- mitigates this issue, yielding more robust models with enhanced generalization in windy conditions.
