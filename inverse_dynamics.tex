Inverse dynamics is a widely used technique within robotics \citep{faber_inverse_2018}; It can be used to estimate the total forces ($X_D,Y_D,N_D$) acting on a ship during a certain motion. The technique can be applied on data from a free model manoeuvring tests or data obtained from a real ship. The total forces are calculated from the kinematics of the manoeuvring model (\autoref{eq:X} to \autoref{eq:N}). These equations require that the full state of the ship is known -- so that data on the position and orientation of the ship as well as the higher states: velocity, and accelerations, are needed.
Only the position and orientation of the ship model where however measured during the model tests.
Therefore the higher states needed to be be estimated with an extended Kalman filter (EKF) -- where the manoeuvring model was used as the predictor \citep{alexandersson_wpcc_2022}.

The estimated inverse dynamics forces was regressed with ordinary least squares (OLS) multiple
linear regression to identify the parameters of a model. In this paper, the inverse dynamics regression was conducted on model test data from: zigzag10/10, and zigzag20/20, to port and starboard.  