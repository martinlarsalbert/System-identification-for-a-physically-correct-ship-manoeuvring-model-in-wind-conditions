ID is a widely used technique within robotics \citep{faber_inverse_2018, haningerNonparametricInverseDynamic2019, mastalliInverseDynamicsMPCNullspace2023, sunHighorderInverseDynamics2023, kurtzInverseDynamicsTrajectory2023}. It can be used to estimate the total forces acting on a ship during motion. The technique can be applied to data from free-model manoeuvring tests or real ship manoeuvres. 
Estimations of the total damping forces can be solved from the manoeuvring model kinematic equations  (\autoref{eq:X}--\autoref{eq:N}). These equations require that the mass, added mass, and full state of the ship are known so that data on the position and orientation of the ship, as well as the higher states, e.g., velocities and accelerations, are known.
However, in the model tests used in this paper, only the position and orientation of the ship model were measured.
The higher states were thus estimated using an extended Kalman filter (EKF), where the manoeuvring model was used as the predictor \citep{alexandersson_system_2022}.

The parameter estimations are defined as a linear regression problem (\autoref{eq:regression})---one for each degree of freedom. 
\begin{equation}\label{eq:regression}
\begin{split}y = \mathbf{X}\gamma + \epsilon\end{split}
\end{equation}
The calculations for the label vector \(y\) and the feature matrix \(\textbf{X}\) differ for the PU and PI models. The PU model has a data-driven rudder model so that the entire damping forces $X_D,Y_D,N_D$ are included in the regression (\autoref{eq:X_H_estimation_abkowitz}). On the other hand, in the PI model, the semi-empirical rudder model is deterministic and is excluded from the regression (\autoref{eq:X_H_estimation_PISM}). 
\begin{equation}
    \label{eq:X_H_estimation_abkowitz}
    y = (\bullet)_D
\end{equation}
\begin{equation}
    \label{eq:X_H_estimation_PISM}
    y = (\bullet)_H = (\bullet)_D - (\bullet)_R
\end{equation}
where \textbullet\ represents the degrees of freedom ($X,Y,N$).
For example, the regression of the surge degree of freedom label \(y\) can be calculated using the ID force (\autoref{eq:label}). 
The feature matrix \(\textbf{X}\) and coefficient vector $\gamma$ are expressed from the model damping force polynomials (\autoref{eq:features} and \autoref{eq:gamma}), where $\delta^2$ and $X_{\delta\delta}$ are removed for the PI model.
\begin{equation}\label{eq:label}
\begin{split}\displaystyle X_D = - X_{\dot{u}} \dot{u}' + \dot{u}' m' - m' r'^{2} x_{G'} - m' r' v'\end{split}
\end{equation}
\begin{equation}\label{eq:features}
\begin{split}\displaystyle X = \left[\begin{matrix}1 & u' & (\delta^{2}) & r'^{2} & v'^{2} & r' v'\end{matrix}\right]\end{split}
\end{equation}
\begin{equation}\label{eq:gamma}
\begin{split}\displaystyle \gamma = \left[\begin{matrix}X_{0}\\X_{u}\\(X_{\delta\delta})\\X_{rr}\\X_{vv}\\X_{vr}\end{matrix}\right]\end{split}
\end{equation}
The hydrodynamic derivatives in the \(\gamma\) vector are estimated with OLS multiple linear regression.
In this regression, the hydrodynamic derivatives are treated as Gaussian random variables, and in the manoeuvring model, they are usually estimated as the mean value of each regressed random variable, the most likely value.

Accurate mass and added mass values are more critical when employing ID in a physics-informed model than in a completely data-driven model.
The data-driven model can give good simulation results even if the mass and added masses are wrong---if the forces are equally wrong---since the erroneous masses cancel in Newton's second law of motion, as in \autoref{eq:Fma} and \autoref{eq:cancel}, where $\epsilon$ is the mass error.
\begin{equation}
    \label{eq:Fma}
    F = (m+\epsilon) \cdot a
\end{equation}
\begin{equation}
    \label{eq:cancel}
    a = \frac{F}{m+\epsilon} = \frac{(m+\epsilon) \cdot a}{m+\epsilon}
\end{equation}
When a deterministic force $F_R$ is introduced in the PI model, however, the acceleration contribution $a_R$ of this force is wrong if the mass is wrong (\autoref{eq:mass_scaling}). Therefore, having correct mass values becomes crucial.
\begin{equation}
    \label{eq:mass_scaling}
    a_R = \frac{F_R}{m+\epsilon}
\end{equation}