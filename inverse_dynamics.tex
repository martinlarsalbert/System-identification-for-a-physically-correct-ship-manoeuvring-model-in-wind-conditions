\subsection{Inverse dynamics and regression}
\label{sec:inverse_dynamics}
Inverse dynamics is a technique to estimate the total forces ($X_D,Y_D,N_D$) acting on a ship during a certain motion. Inverse dynamics can for instance be used to estimate the forces from a manoeuvring model test with a free model or motions measured on a real ship. The total forces are calculated from the kinematics of the manoeuvring model (\autoref{eq:X} to \autoref{eq:N}). These equations require the full state of the ship, so that the position and orientation of the ship as well as the higher states: velocity, and accelerations must be known. 
Only the position and orientation where measured in the model tests used in this paper, so that the higher states must be estimated. This estimation was done with an Extended Kalman filter (EKF) for this paper.

The estimated inverse dynamics forces can be regressed in a similar way as the VCT data (\autoref{sec:VCT_regression}), in a "inverse dynamics regression".  