The parameters within the reference model are identified by regression on a VCT dataset, as conducted similarly in \citet{marimon_giovannetti_effects_2020}. The VCT involves CFD, where the state was varied according to \autoref{tab:vct_variations}. For instance, in the VCT circle test, yaw rate $r$ is varied while surge velocity $u$ is constant, drift angle is zero ($v=0$), rudder angle is zero, and the propeller is at the self-propulsion point ($\eta_0=1$).

The regression is conducted with ordinary least-squares (OLS) multiple linear regression. However, instead of regressing all parameters simultaneously, the regression is divided into many sub-regressions to reduce the multicollinearity. The regressions are defined as a step-wise process in a regression pipeline. 
The regression pipeline works similarly to solving an equation system. For instance, ${X_{vv}}'$ is determined from the drift angle variation before the circle + drift variation is regressed so that ${X_{vv}}'$ can be used as a known value when ${X_{vr}}'$ is determined. The sub-regressions must thus be performed in the correct order. 
%The regression pipelines of the PU model (\autoref{tab:vct_pipeline_abkowitz}), and the PI model (\autoref{tab:vct_pipeline_semiempirical}) are different. 

The regression pipeline of the reference model is shown in \autoref{tab:vct_pipeline_semiempirical}, where some of the VCT test types (\autoref{tab:vct_variations}) are regressed and others are used for validation. Therefore, the semi-empirical rudder is treated as a deterministic model and is not included in the regression. The regression is instead performed on the hull forces $X_H,Y_H,N_H$ obtained from the VCT calculations. However, the total sway force $Y_D$ and total yawing moment $N_D$ are used to determine the rudder hull interaction coefficients ${a_H}',{x_H}'$. 
\begin{table}[h!]
    \centering
    \caption{Parameter variations in virtual captive tests, where a fixed value is indicated by -, $\sim$ means the value varies, and 0 means the variable is zero.}
    \label{tab:vct_variations}
    \pgfplotstabletypeset[col sep=comma,
    %columns={Test type,$u$,$v$,$r$,$\delta$,$\eta_0$},
    columns/Test type/.style={string type, column type=l},
    columns/$u$/.style={string type},
    columns/$v$/.style={string type},
    columns/$r$/.style={string type},
    columns/delta/.style={string type},
    columns/$eta_0$/.style={string type},
    %columns/SI unit/.style={string type},
    %columns/Physical quantity/.style={string type},
    %columns/Denominator/.style={string type},
    %column type=l,	% specify the align method
    %every head row/.style={before row=\hline,after row=\hline},	% style the first row
    %every last row/.style={after row=\hline},	% style the last row
    every head row/.style={before row=\hline,after row=\hline},
    every last row/.style={after row=\hline}
    ]{tables/virtual_captive_tests.csv"}
\end{table}
%\begin{table}[h!]
%    \centering
%    \caption{Pipeline for the VCT regression of the PU model}
%    \label{tab:vct_pipeline_abkowitz}
%    \pgfplotstabletypeset[col sep=comma, string type, column type=l
%    ]{tables/method_VCT_regression.regression_pipeline_abkowitz.csv"}
%\end{table}
\begin{table}[h!]
    \centering
    \caption{Pipeline for the VCT regression of the PI model}
    \label{tab:vct_pipeline_semiempirical}
    \pgfplotstabletypeset[col sep=comma, string type, column type=l,
    every head row/.style={before row=\hline,after row=\hline},
    every last row/.style={after row=\hline}
    ]{tables/method_VCT_regression_abkowitz.regression_pipeline_semiempirical.csv"}
\end{table}