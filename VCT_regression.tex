\subsection{Virtual captive tests regression}
\label{sec:VCT_regression}
The Abkowitz model and the Semi-empirical model can be identified by regressing their parameters on virtual captive test (VCT) where the state has been varied according to \autoref{tab:vct_variations}. For instance in the VCT Circle yaw rate $r$ is varied while: surge velocity $u$ is constant, no drift angle ($v=0$), no rudder angle, and the propeller is at the self propulsion point ($\eta_0=1$).  
\begin{table}[h!]
    \centering
    \caption{Parameter variations in virtual captive tests where fixed value is indicated by -, varying value by $\sim$ and 0 means that the variable is zero.}
    \label{tab:vct_variations}
    \pgfplotstabletypeset[col sep=comma,
        %columns={Test type,$u$,$v$,$r$,$\delta$,$\eta_0$},
        columns/Test type/.style={string type, column type=l},
        columns/$u$/.style={string type},
        columns/$v$/.style={string type},
        columns/$r$/.style={string type},
        columns/delta/.style={string type},
        columns/$eta_0$/.style={string type},
        %columns/SI unit/.style={string type},
        %columns/Physical quantity/.style={string type},
        %columns/Denominator/.style={string type},
        %column type=l,	% specify the align method
        %every head row/.style={before row=\hline,after row=\hline},	% style the first row
        %every last row/.style={after row=\hline},	% style the last row
    ]{tables/virtual_captive_tests.csv"}
\end{table}

\begin{table}[h!]
    \centering
    \caption{Pipeline for the VCT regression of the Abkowitz model}
    \label{tab:vct_variations}
    \pgfplotstabletypeset[col sep=comma, string type, column type=l
    ]{tables/method_VCT_regression.regression_pipeline_abkowitz.csv"}
\end{table}

\begin{table}[h!]
    \centering
    \caption{Pipeline for the VCT regression of the Semi-empirical model}
    \label{tab:vct_variations}
    \pgfplotstabletypeset[col sep=comma, string type, column type=l
    ]{tables/method_VCT_regression_abkowitz.regression_pipeline_semiempirical.csv"}
\end{table}