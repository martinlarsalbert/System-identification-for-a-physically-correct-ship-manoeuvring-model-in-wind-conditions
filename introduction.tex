%________________________________________
%CaRS Move I "Establish a territory" (Situation):
% * Important area
% * Introducing and reviewing items of previous  research <--(ToDO)
Modern ships gather vast amounts of kinematic data, including high-precision GPS, accelerometers, and inclinometers, etc. Similarly to the emergence of the Internet-of-Things (IoT) we can now talk about Internet of Ships (IoS) \citep{liu_internet_2016}, where safety enhancements, route planning and optimization, energy efficiency, automatic berthing and autonomous shipping are some of the emerging applications \citep{aslam_internet_2020}.
Predictive modeling can be used for IoS to make predictions about future events or outcomes based on the gathered historical data.
Predictive modeling using machine learning (ML) techniques is commonly employed for systems with unknown physics, making ML an adaptable choice for forecasting. However, in cases where prior knowledge of a system's physics and structure is available, such as ship manoeuvring, system identification emerges as a valuable alternative. By leveraging established knowledge, system identification models establish causal relationships between variables, which is an important aspect to enable optimization of the ship operation.
A lot of research has been devoted to describe the ship manoeuvring dynamics with system based manoeuvring simulation models such as: \citet{abkowitz_ship_1964,nomoto_steering_1957,norrbin_theory_1971,yasukawa_introduction_2015}, and others.

Captive model tests is the classical approach to identify the parameters within these models. This approach is not practically possible for the full scale ships, however; Instead, virtual captive tests (VCT) -- which are essentially simulated captive tests with computational fluid dynamics (CFD) -- has emerged as an interesting option for the full scale ship predictions. 
The CFD requires a complete understanding of the system. Acquiring such knowledge may be possible for some cases, but it is not practical for the complex environment and nonlinearities of a ship operating at sea \citep{miller_ship_2021}.
Wind, waves, and currents add uncertainty to the modelling in the deep sea. Water depth and the bank effect add uncertainty in coastal areas \citep{\nielsen_machine_2022}.
Even if the sea is flawlessly modelled, long-term predictions with high accuracy will be exposed to deterministic chaos \citep{lorenz_deterministic_1963}.

, where system identification based on the measured trajectory of the ship is instead    

% * Introducing and reviewing items of previous  research <--(ToDO)
The input variables of the manoeuvring model are often strongly linearly dependent; This multicollinearity is a well known issue in parameter identification that may lead to parameter drift and poor generalization; The parameters are thus mathematically correct but physically incorrect \citep{luo_parameter_2016}. 
Simplifying the model -- when possible -- and thereby reducing the number of parameters, is perhaps the most effective way to mitigate the multicollinearity.
Some other possible remedies are: the difference method \citep{luo_parameter_2016}, principal component analysis (PCA), and partial least squares regression \citep{jian-chuan_parametric_2015}. 

\begin{itemize}
    \item Identification on zigzag manoeuvres
\end{itemize}

%
%________________________________________
%CaRS Move II "Establish a niche" (Problem):
% * counter claim?
% * gap?
% * question       <------
% * continuation?
\begin{itemize}
    \item Multicollinearity
    \item Generalization
\end{itemize}
The correctness of identified models can be mathematical, yielding favorable results on training and test datasets, yet lacking physical accuracy. This discrepancy becomes evident when a model trained in calm water conditions is exposed to wind during maneuvers. This paper explores how models with varying levels of physical correctness respond to such scenarios.

%
%________________________________________
%CaRS Move III "Occupy the niche" (Solution/Evaluation):
% * outline purpose?
% * list research questions?
% * announce principal findings?            <--
% * stating the value of present research?
% * article structure?                      <--
To evaluate the physical correctness of identified models, two unique datasets from a wind-powered pure car carrier (WPCC) are employed. Data from virtual captive tests (VCT), calculated via computational fluid dynamics (CFD) at various steady-state drift angles, yaw rates, and rudder angles, establish the physically correct kinetics. System identification is conducted through inverse dynamics regression (IDR) on a second dataset containing a series of manoeuvring model tests with a free model. The generatlization of the models into an idealized wind state is also studied. 

% Old version:
Prediction of ship maneuverability, which relies on accurate mathematical models to simulate, e.g., Turning Circle Maneuver, Zig-zag Maneuver, etc., is essential for both ship design and intelligent ship navigation. The mathematical ship maneuverability models require inputs of hydrodynamic forces and moments acting on the ship hull, commonly known as “hydrodynamic coefficients”, “maneuvering coefficients” or “maneuvering derivatives” in non-dimensional form. Various approaches have been proposed to derive those non-dimensional hydrodynamic coefficients by different parameter identification methods, such as simple regression, SVM, Kalman filter, and inverse dynamics, such as Tongtong et al. (2020), Zou et al. (2022), Alexandersson et al. (2021, 2022, 2023), ... In Tongtong et al. (2021) and Alexandersson et al. (2022), different mathematical models have been investigated to describe the most accurate and robust ship maneuvering simulations, also referred to the system identification methods.

However, it should be noted that most research activities aimed to identify a ship's maneuverability mathematical models are built on a ship's simulation data, i.e., a type of inverse engineering to find the model used to simulate the ship maneuvering data. While for practical application, the only reliable source of hydrodynamic coefficients is a model test (ABS 2006), which should also be validated later by full scale test /sea trial). Therefore, several knowledge gaps should be investigated to researched to close the gap between purely simulation-based PIM and generic application of the identified MM model from model to actual usage:
\begin{enumerate}
    \item the parameter identification method based on the simulation data can effectively estimate the hydrodynamic coefficients since the exact "psychical" mathematical model is assumed known. In real cases/applications, the mathematical model is just the simplification of real physics (from VCT, model test, sea trials, or full-scale tests), while the complexity of hydrodynamics involved in vessel maneuvering will greatly challenging the simplified empirical mathematical methods without carefully considering the actual physics inside hydrodynamic coefficients and their interactions (ABS 2006).
    \item if some "actual" data (not simulation data) is used to identify the mathematical model, the PIT may be able to give a good prediction of tested maneuvering trajectories, but the identified mathematical maneuverability model may lack of generalization. It means that the model can only be used to predict the ship's maneuverability under the test scenarios, because the mathematical models cannot capture physics in generic maneuvering scenarios.
    \item to make the method worse, when using non-simulated data for system identification of maneuverability models, the identified model may give wrong physics, i.e., non-physical hydrodynamic forces regressed from the model. Obviously, for this case, the mathematical model identified based on tests from Turning Circle Maneuver can be only reliably used to predict a ship's maneuverability under turning circle tests but not for the zig-zag tests, and vice verse. In some special case, the prediction can be limited to replicating similar test conditions, such as zig-zag tests at certain angles.
    \item the prediction of ship maneuverability from the mathematical model (established from ideal conditions such as model tests) is required to be validated/verified by "real" environment conditions, such as sea trials or full-scale tests. The drift caused by wind/current that are not considered in the mathematical maneuverability model will cause completely failure of the identified model applied for real prediction cases.
\end{enumerate}

Therefore, this study aims to propose a holistic approach to integrate some physical hydrodynamic terms in the mathematical maneuverability model for a more generic ship maneuverability system identification. This approach can make use of test data in ideal conditions to establish a physics-guided mathematical maneuverability model, which can predict a ship's maneuverability under more generic maneuvering scenarios.
%% Literature review here...

% Background...
% * Other people are identifying standard manoeuvres
% * 
The present paper highlights the difficulties in identifying a pure mathematical manoeuvring model -- such as the Abkowitz model -- from calm water manoeuvring test motions for a model that should generalize to have physically correct hydrodynamics also in wind conditions. 

% Objective...
% Use VCT CFD to validate the physics
In order to mitigate these difficulties, more prior knowledge about the rudder hydrodynamics has been added to the model. A new semi-empirical rudder model is therefor introduced in this paper, based on semi-empirical formulas from the literature. The identified models with the pure mathematical structure and the introduced semi-empirical rudder are assessed with: manoeuvring model tests in calm water, CFD calculated virtual captive tests (VCT), and an idealized wind condition.

In this paper the manoeuvring models are first introduced in \autoref{sec:theory} -- this is a reference section that may be skipped during the first read of this paper -- followed by methodology in \autoref{sec:methodology} -- about the parameter identification techniques; the case study ship and datasets are then described in \autoref{sec:case_study}, followed by results (\autoref{sec:results}), discussion (\autoref{sec:discussion}), and conclusions (\autoref{sec:conclusions}).
