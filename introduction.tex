%________________________________________
%CaRS Move I “Establish a territory” (Situation):
% * Important area
% * Introducing and reviewing items of previous  research <--(ToDO)
Ship dynamics predictive models have a wide range of applications, e.g., safety enhancements, and route planning and optimization, autonomous shipping \citep{aslam_internet_2020}.
Ship manoeuvring is a sub-field of ship dynamics with well-established system-based models such as \citet{abkowitz_ship_1964,nomoto_steering_1957,norrbin_theory_1971}, and the MMG (manoeuvring modeling group) model \citep{yasukawa_introduction_2015}.

The captive model test is the classical method to identify the parameters within these models. However, for full-scale ships, this method is not practical. Computational fluid dynamics (CFD) with either unsteady Reynolds averaged Navier--Stokes (URANS) or steady Reynolds averaged Navier--Stokes (RANS) computations in virtual captive tests (VCTs) has emerged as an interesting option \citep{liu_predictions_2018,li_ship_2022}.
CFD requires a complete understanding of the system, which is straightforward for some simplified scenarios, but large modeling uncertainties from wind, wave, and current are expected when applied in a complex sea environment \citep{miller_ship_2021}. 
Even if the sea is flawlessly modeled, long-term predictions with high accuracy are exposed to deterministic chaos \citep{lorenz_deterministic_1963}.
With the other drawbacks of CFD in manoeuvring---such as high computational costs---data-driven models have become an attractive alternative or complement, with an increased number of publications in the past 10--15 years, %(shown in Fig.\ref{fig:pub_overview})
especially within the field of autonomous ships \citep{ahmed_survey_2023}, where predicting  ship trajectories is critical to avoid collisions. 

%Relevant paper keywords and their connections to the most recent research activities related to ship maneuvering modeling are presented in \autoref{fig:pub_overview}, where the circle size indicates how often the keywords occurred in the research articles. These papers claim their proposed regression-based methods can estimate the hydrodynamic coefficients of a ship’s maneuvering system accurately. However, most of the regression methods assume the exact “physical” model is known, i.e., the regression is based on “simulation” data, as indicated in \autoref{fig:pub_overview}, where the large circles of “simulation” and “maneuvering simulation” are directly connected with others.
%Most of the simulations (as shown in the keywords) were based on the model in \citet{fossen_handbook_2021}, such as the Gaussian process regression method in \citet{xue_system_2020} and the support vector machine method in \citet{wang_identification_2019} and \citet{wang_parameter_2021}. Recently, the support vector machine and neural network methods have also been used to identify the parameters of ship dynamic models using data from benchmark model tests with well-known ship maneuverability (e.g., SIMMAN, SHOPERA), as in \citet{wang_kernel-based_2020} and \citet{wakita_neural_2021}. The identified maneuvering models were validated by comparing their capabilities for trajectory prediction with either simulated or well-controlled model test results. However, the model accuracies in describing the ship’s kinetics, e.g., various forces acting on ships, was rarely mentioned. Nevertheless, such dynamic performance is essential to determine model feasibility for the application of a ship’s operation in real sea environments, i.e., with drifting caused by wind and current.
%%
%\begin{figure}[h]
%  \includegraphics[width=\textwidth]{figures/keywords.png}
%  \caption{Research topics within the field of ship maneuverability system identification.}
%  \label{fig:pub_overview}
%\end{figure}
%%
%fication_1976} to develop a linear manoeuvring model that utilized manually recorded data in 1969 aboard the Atlantic Song freighter with Kalman filter (KF) and maximum likelihood estimation. 
%a_system_2015}, a 3 degree of freedom model by \citet{shi_identification_2009}, and a recursive EKF by \citet{alexandersson_system_2022}.
%KF for handling nonlinear systems, was used in \citet{revestido_herrero_two-step_2012}.
%\citet{zhu_parameter_2017}, and \citet{wang_parameter_2021}. 

%
%________________________________________
%CaRS Move II “Establish a niche” (Problem):
% * counter claim?
% * gap?
% * question       <------
% * continuation?
The regressors of the data-driven ship manoeuvring models are often strongly linearly dependent. 
In the beginning of a turning manoeuvre, forces are only generated by the rudder; But very soon after, the ship will also have a yaw rate and drift angle, so that force is also generated at the hull surface. The total force acting on the ship is often used as the dependent variable in the regression, especially when the force from the rudder cannot be measured or estimated. The dependent variable is thus the sum of hull and rudder force, so that hull and rudder coefficients become strongly linearly dependent in the manoeuvre regression.
This multicollinearity is a well-known issue in parameter identification that may lead to parameter drift and poor generalization. The parameters are thus mathematically correct but physically incorrect \citep{luo_parameter_2016-1}. 
An example of generalization that is discussed in the present paper is when a model identified on calm water data is exposed to wind, where a drift angle is needed to maintain a straight course. This wind state is rare in calm water manoeuvring tests, where the drift angle is almost exclusively accompanied by yaw rate (see \autoref{fig:phase_portrait}); 
Thus, the regression has problems when distinguishing between these quantities.
% Wengang fixes a better figure here...
%This drifting (side forces) is rarely investigated as shown in %\autoref{fig:pub_overview}.
%
\begin{figure}[h]
  \centering
  \includesvg{figures/multicollineraity.multicollinearity.svg}
  \caption{Phase portrait where the combination of drift angle and yaw rate is shown for zigzag10/10 and zigzag20/20 wPCC model tests.}
  \label{fig:phase_portrait}
\end{figure}
Using more informative data or simplifying the model when possible, thereby reducing the number of parameters, is a commonly researched approach to mitigating multicollinearity. Other possible remedies are the difference method \citep{luo_parameter_2016-1}, principal component analysis (PCA), and partial least-squares regression \citep{jian-chuan_parametric_2015}. 
%________________________________________
%CaRS Move III “Occupy the niche” (Solution/Evaluation):
% * outline purpose?                        <--
% * list research questions?                <--
% * announce principal findings?            
% * stating the value of present research?
% * article structure?                      <--

However, more remedies are needed. Therefore, this paper proposes a physics-informed manoeuvring model (PI model), which features a new semi-empirical rudder model to estimate the rudder forces, so that the rudder and hull forces can be separated in the regression, which reduces the multicollinearity.

This paper investigates whether the PI model is more physically correct than a physics-uninformed model (PU model) when they are both identified from zigzag model tests. It is also studied how the lack of physical correctness may impact the model generalization. The sensitivity to small changes in the data is also studied for the PI and PU models.
To evaluate the physical correctness, a wind-powered pure car carrier (wPCC) test case is studied. 
This ship has much larger rudders than conventional ships---to improve the sailing performance---
which increases the demand for physically correct rudder modeling. The generalization of the models is studied in an idealized wind state.

A brief description of the workflow of this research is shown in \autoref{fig:methodology}.
The PI and PU models are identified on free sailing model tests \citep{alexandersson_system_2022,alexandersson_wpcc_2024} via inverse dynamics (ID) \citep{faber_inverse_2018} and regression. To assess the physical correctness, a reference model is established, where the PI model is instead identified on a VCT dataset. This reference model, based on CFD, is assumed to be a sufficiently correct representation of the ship's physics.
Verification and comparisons between the models are carried out on the free sailing model tests. The generalization of the models is then studied on an idealized wind condition.
%
\begin{figure}[h]
  \centering
  %\includesvg[width=\columnwidth, pretex=\scriptsize, height=12cm]{figures/methodology2.svg}
  \includesvg[pretex=\centering\fontsize{7.5}{8}]{figures/methodology2.svg}
  \caption{Research workflow.}
  \label{fig:methodology}
\end{figure}

The remainder of this paper is organized as follows: The proposed PI model is first presented together with the PU 
 model in \autoref{sec:ship_models}, while mathematical details of the models are given in the appendix. 
\autoref{sec:methodology} describes the developed methodology framework to identify parameters within the PI model, including VCT-- and ID--regressions. The case study ship is briefly described in \autoref{sec:case_study}, along with known parameters of the ship’s manoeuvring model. \autoref{sec:results} provides the results for the wPCC. Results for the KVLCC2 test case are also presented, followed by key conclusions of this research in \autoref{sec:conclusions}.