\section{Introduction}
\label{sec:introduction}

%% Literature review here...

%A captive model test is the classic type of experiment used in ship hydrodynamics to determine the hydrodynamic coefficients of a ship’s hull, rudder, and propeller. The test is conducted by towing a model of the ship in a towing tank or a maneuvering basin. The model is held stationary while the water is made to flow around it. The hydrodynamic forces and moments acting on the model are measured and used to calculate the coefficients.

The present paper highlights the difficulties in identifying a pure mathematical manoeuvring model -- such as the Abkowitz model -- from calm water manoeuvring test motions for a model that should generalize to have physically correct hydrodynamics also in wind conditions. In order to mitigate these difficulties, more prior knowledge about the rudder hydrodynamics is added to the model. A new semi-empirical rudder model is therefor introduced, based on semi-empirical formulas from the literature. The identified models with the pure mathematical structure and the introduced semi-empirical rudder are then assessed with: manoeuvring model tests in calm water, CFD calculated virtual captive tests (VCT), and an idealized wind condition.

In this paper the manoeuvring models are first introduced in \autoref{sec:theory} -- this is a reference section that may be skipped during the first read of this paper -- followed by methodology in \autoref{sec:methodology} -- about the parameter identification techniques; the case study ship and datasets are then described in \autoref{sec:case_study}, followed by results (\autoref{sec:results}), discussion (\autoref{sec:discussion}), and conclusions (\autoref{sec:conclusions}).
