\begin{itemize}
    \item A new semi-empirical rudder model has been proposed, combining existing semi-empirical formulas together with some new enhancements: the rudder area covered or uncovered by the propeller slip stream are modelled separately, a new way to model the effect of reduced lift due to gap between rudder and rudder horn has also been proposed.     

    \item All of the identified models in this paper can be considered as mathematically correct, since they all predicted the model tests with satisfactory agreement.
    
    \item The Reference model predicted the VCT with good accuracy and can therefore be considered as the most physically correct model.   
    %
    \item As a logical consequence of this, the mathematical Abkowitz model is considered as physically incorrect, since it predicted very different forces and moments; The regression seems to have made an incorrect decomposition between hull-, and rudder forces, as well as drift angle-, and yaw rate dependent coefficients.  
    %
    \item The physics informed model on the other hand, predicted very similar forces and moments for the model tests and can therefore considered as a more physically correct.
    %
    \item Potential problems with the incorrect force decomposition of the Abkowitz model was shown in the lack of generalization of an artificial wind state, where the forces and moments had very large errors. 
    %
    \item The introduction of a semi-empirical rudder model seems to have guided the identification towards a more physically correct model, which generalizes better from calm water zigzag tests to wind conditions. 
    
    \item The procedure presented in this paper is one way to mitigate the well known issues with high multicollinearity within the ship manoeuvring models.
\end{itemize}