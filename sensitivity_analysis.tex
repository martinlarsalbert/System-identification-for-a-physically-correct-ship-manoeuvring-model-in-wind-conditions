The sensitivity of the ID regression was studied for the PU and PI models, to see how sensitive they are to small changes in the data.
The model test data was filtered with EKF:s with different covariance matrices of the process noise \textbf{Q} and observation noise \textbf{R}. The observation noise \textbf{R} was varied by changing the signal to noise ratio (SNR) according to \autoref{eq:SNR}, where a larger SNR gives as smaller \textbf{R} so that the EKF thereby relies more on the model test data than the predictor model.
\begin{equation}
    \label{eq:SNR}
    \mathbf{R} = \frac{\mathbf{Q}}{SNR}
\end{equation}
Filtered data from one of the zigzag test with the covariance varied by SNR=0.1, 1, and 10 are shown in \autoref{fig:SNR_sensitivity.accelerations}. SNR=10 relies more on the data and therefore contains more of the measurement noise.
\begin{figure}[h]
    \begin{center}
        \includesvg{figures/SNR_sensitivity.accelerations.svg}
        \caption{Kalman filtered yaw and sway accelerations with varying covariance, indicated by the signal to noise ratio (SNR).}
        \label{fig:SNR_sensitivity.accelerations}
    \end{center}
\end{figure}