The parameter drift of the PU and PI models was studied in a sensitivity analysis, to see how sensitive the models are to small changes in the data.
The model test data was filtered with EKF:s with different covariance matrices of the process noise \textbf{Q} and observation noise \textbf{R}. The observation noise \textbf{R} was varied by changing the signal to noise ratio (SNR) according to \autoref{eq:SNR}, where a larger SNR gives a smaller \textbf{R}, so that the EKF thereby relies more on the model test data than the predictor model.
\begin{equation}
    \label{eq:SNR}
    \mathbf{R} = \frac{\mathbf{Q}}{SNR}
\end{equation}
Accelerations of the filtered data from one of the zigzag tests with the covariance varied by SNR=0.1, 1, and 10, are shown in \autoref{fig:SNR_sensitivity.accelerations}. SNR=10 relies more on the data and therefore contains more of the measurement noise.
\begin{figure}[h]
    \begin{center}
        \includesvg{figures/SNR_sensitivity.accelerations.svg}
        \caption{Kalman filtered yaw and sway accelerations with varying covariance, indicated by the signal to noise ratio (SNR).}
        \label{fig:SNR_sensitivity.accelerations}
    \end{center}
\end{figure}
Identified hull coefficients from the data filter variations are shown in \autoref{tab:pivot}.
\begin{table}[h]
    \centering
    \caption{Identified hull coefficients for the PI and PU models identified on EKF filtered data with varying signal to noise ratio (SNR).}
    \label{tab:pivot}
    \pgfplotstabletypeset[col sep=comma, column type=r,
        columns/Coefficient/.style={column type=c,string type},
    every head row/.style={before row=\hline,after row=\hline},
    every last row/.style={after row=\hline}
    ]{tables/SNR_sensitivity.pivot.csv}
\end{table}
Large differences was observed for the PU model, especially in the identified yaw coefficients ${N}'_r,{N}'_v$ (also shown in \autoref{fig:SNR}) where ${N}'_v$ even has the wrong sign for SNR=0.1, and SNR=1.0. The identified values of the PI model is more stable during the variations. Th PI model is therefore more robust and less sensitive to small variations in the data.
\begin{figure}
    \centering
    \begin{subfigure}[b]{0.49\textwidth}
        \centering
        \includesvg[width=\textwidth]{figures/SNR_sensitivity.Nr.svg}
        \caption{${N}'_r$.}
        \label{fig:SNR_sensitivity.Nr}
    \end{subfigure}
    \vfill
    \begin{subfigure}[b]{0.49\textwidth}
        \centering
        \includesvg[width=\textwidth]{figures/SNR_sensitivity.Nv.svg}
        \caption{${N}'_v$.}
        \label{fig:SNR_sensitivity.Nv}
    \end{subfigure}
    
    \caption{Identified yaw hull coefficients for the PI and PU models identified on EKF filtered
data with varying signal to noise ratio (SNR).}
    \label{fig:SNR}
\end{figure}