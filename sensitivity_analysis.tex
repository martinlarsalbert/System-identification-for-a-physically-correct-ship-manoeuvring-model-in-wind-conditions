The sensitivity of the ID regression was studied for the PU and PI models, to see how sensitive they are to small changes in the data.
The model test data was filtered with EKF:s with different covariance matrices of the process noise \textbf{Q} and observation noise \textbf{R}. The observation noise \textbf{R} was varied by changing the signal to noise ratio (SNR) according to \autoref{eq:SNR}, where a larger SNR gives as smaller \textbf{R}, so that the EKF thereby relies more on the model test data than the predictor model.
\begin{equation}
    \label{eq:SNR}
    \mathbf{R} = \frac{\mathbf{Q}}{SNR}
\end{equation}
Filtered data from one of the zigzag test with the covariance varied by SNR=0.1, 1, and 10 are shown in \autoref{fig:SNR_sensitivity.accelerations}. SNR=10 relies more on the data and therefore contains more of the measurement noise.
\begin{figure}[h]
    \begin{center}
        \includesvg{figures/SNR_sensitivity.accelerations.svg}
        \caption{Kalman filtered yaw and sway accelerations with varying covariance, indicated by the signal to noise ratio (SNR).}
        \label{fig:SNR_sensitivity.accelerations}
    \end{center}
\end{figure}
The PI and PU models were identified on the model test data filter variations. The identified hull coefficients are shown in \autoref{tab:pivot}.
\begin{table}[h]
    \centering
    \caption{Identified hull coefficients for the PI and PU models identified on EKF filtered data with varying signal to noise ratio (SNR).}
    \label{tab:pivot}
    \pgfplotstabletypeset[col sep=comma, column type=r,
        columns/Coefficient/.style={column type=c,string type},
    every head row/.style={before row=\hline,after row=\hline},
    every last row/.style={after row=\hline}
    ]{tables/SNR_sensitivity.pivot.csv}
\end{table}
\begin{figure}
    \centering
    \begin{subfigure}[b]{0.49\textwidth}
        \centering
        \includesvg[width=\textwidth]{figures/SNR_sensitivity.Nr.svg}
        \caption{Yaw rates.}
        \label{fig:vct_circle}
    \end{subfigure}
    \vfill
    \begin{subfigure}[b]{0.49\textwidth}
        \centering
        \includesvg[width=\textwidth]{figures/SNR_sensitivity.Nr.sv}
        \caption{Drift angles.}
        \label{fig:vct_drift_angle}
    \end{subfigure}
    
    \caption{Comparison between the VCT data ($+$) and predictions with the reference model (lines) (expressed in the prime system $\times 1000$) for total damping force $D$ (red), hull force $H$ (green), and rudder forces $R$ (blue)}
    \label{fig:vct}
\end{figure}