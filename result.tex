The hydrodynamic derivatives within the PI and PU models were identified with inverse dynamics regression (\autoref{sec:inverse_dynamics}) and the reference model was identified by regression on a VCT dataset (\autoref{sec:VCT_regression}). The identified values are shown in \autoref{tab:parameters}. 

In order to establish the the reference model, a comparison with the underlying VCT data is first presented in \autoref{sec:result_VCT}. The reference model is then used to assess the physical correctness of the identified PI and PU models in \autoref{sec:result_MDL}. 
The generalization is then studied on an idealized wind state in \autoref{sec:idealized_wind_state}.
The parameter drift is studied in a sensitivity analysis in \autoref{sec:sensitivity}. 
Lastly, results from an additional test case is also briefly studied in \autoref{sec:result_KVLCC2_HMRI} to see how the PI model behaves for a completely different ship type.

\begin{table}[h]
    \centering
    \caption{Identified hull coefficients in prime system units.}
    \label{tab:parameters}
    \pgfplotstabletypeset[col sep=comma, column type=r,
        columns/Coefficient/.style={column type=c,string type},
    every head row/.style={before row=\hline,after row=\hline},
    every last row/.style={after row=\hline}
    ]{tables/result_models.parameters.csv}
\end{table}
