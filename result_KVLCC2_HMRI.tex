A similar investigation was also conducted for the KVLCC2 test case. The PI, and PU models where identified with ID regression on zigzag10/10, and zigzag20/10 model tests -- to port and starboard. The model test data was measured from the Hamburg ship model basin (HSVA) for the SIMMAN2008 workshop \citep{stern_experience_2011}.
The MMG wake model was added with coefficient values according to \citet{yasukawa_introduction_2015}.
The identified prediction models were compared with captive model tests (CMT) measured at the Hyundai Maritime Research Institute (HMRI) for the SIMMAN2014 workshop \citep{ittc_final_2017}. A comparison for the drift angle variation is shown in \autoref{fig:KVLCC2_HMRI}. 
Similar trends to the wPCC in the idealized wind state can be observed; The PI model predicts total sway force, and yawing moment that are closer to the CMT than the PU model, as seen in \autoref{fig:KVLCC2_HMRI_D}.
This can be explained by the PU models inability to identify the correct rudder forces, as seen in \autoref{fig:KVLCC2_HMRI_R}.
%
\begin{figure}
    \centering
    \begin{subfigure}[b]{0.49\textwidth}
        \centering
        \includesvg[width=\textwidth]{figures/results_kvlcc2.KVLCC2_HMRI_D}
        \caption{Total forces.}
        \label{fig:KVLCC2_HMRI_D}
    \end{subfigure}
    \hfill
    \begin{subfigure}[b]{0.49\textwidth}
        \centering
        \includesvg[width=\textwidth]{figures/results_kvlcc2.KVLCC2_HMRI_R}
        \caption{Rudder forces.}
        \label{fig:KVLCC2_HMRI_R}
    \end{subfigure}
    \caption{KVLCC2 forces and moments (expressed in prime system times 1000): HMRI captive model tests (CMT -- black), PI (green), and PU (red).}
    \label{fig:KVLCC2_HMRI}
\end{figure}