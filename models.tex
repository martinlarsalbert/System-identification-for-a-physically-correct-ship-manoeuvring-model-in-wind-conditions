The ship's kinematics are expressed amidship in a ship fixed reference frame, rotated around the Earth fixed axis $x_0$ by the heading angle \(\Psi\). Forces and motions are expressed in the surge $X$, sway $Y$, and yaw $N$ degrees of freedom as shown in \autoref{fig:reference_frames}. 
$X_D,Y_D,N_D$ and $u,v,r$ are the damping forces, moment and velocities in $X,Y,N$, respectively.
The total velocity is defined by the magnitude \(V\) and drift angle \(\beta\). The rudder angle of the two rudders is denoted by \(\delta\).
The kinematics are described by \autoref{eq:X}--\autoref{eq:N}. 
These equations have the added mass coupling terms, $Y_{\dot{r}}$ and $N_{\dot{v}}$, which are neglected in the MMG model. Subscript D refers to the damping forces and moment \citep{fossen_handbook_2021}, which can be interpreted as the total hydrodynamic force acting on the ship due to the velocity through water.
%
\begin{figure}[h]
    \centering
    \includesvg{figures/reference_frames.svg}
    \caption{Reference frames}
    \label{fig:reference_frames}
\end{figure}
%
\begin{equation}
    \label{eq:X}
    m \left(\dot{u} - r^{2} x_{G} - r v\right) = X_{D} + X_{\dot{u}} \dot{u}
\end{equation}
%
\begin{equation}
    \label{eq:Y}
    m \left(\dot{r} x_{G} + \dot{v} + r u\right) = Y_{D} + Y_{\dot{r}} \dot{r} + Y_{\dot{v}} \dot{v}
\end{equation}
%
\begin{equation}
    \label{eq:N}
    I_{z} \dot{r} + m x_{G} \left(\dot{v} + r u\right) = N_{D} + N_{\dot{r}} \dot{r} + N_{\dot{v}} \dot{v}
\end{equation}
The damping forces and moments are expressed in a modular way, as shown in \autoref{eq:X_D}--\autoref{eq:N_D} and \autoref{fig:force_model},
% Components:
\begin{equation}
    \label{eq:X_D}
    X_{D} = X_{H} + X_{P} + X_{R}
\end{equation}
%
\begin{equation}
    \label{eq:Y_D}
    Y_{D} = Y_{H} + Y_{P} + Y_{R} + Y_{RHI}
\end{equation}
%
\begin{equation}
    \label{eq:N_D}
    N_{D} = N_{H} + N_{P} + N_{R} + N_{RHI}
\end{equation}
%
\begin{figure}[h]
    \centering
    \includesvg[width=4cm]{figures/force_model.svg}
    \caption{Modular force components.}
    \label{fig:force_model}
\end{figure}
where subscripts $H$, $P$, $R$, and $RHI$ represent contributions from the hull, propellers, rudders, and rudder hull interaction, respectively. The rudder hull interaction having its own element is a difference from the MMG model.
% Hull:

The hull forces are expressed with the same polynomials as the MMG model, except that the ${X_{vvvv}}'$ coefficient is omitted and an additional resistance term ${X_u}'$ is added to allow for a more nonlinear resistance (see \autoref{sec:hull}). The nonlinear resistance is possible because of the use of perturbed velocity (see \autoref{sec:prime_system}).
%Propellers:
The total twin screw propeller forces are expressed as
\begin{equation}
    \label{eq:X_P}
    X_{P} = X_{P port} + X_{P stbd}
\end{equation}
%
\begin{equation}
    \label{eq:Y_P}
    Y_{P} = 0
\end{equation}
%
\begin{equation}
    \label{eq:N_P}
    N_{P} = N_{P port} + N_{P stbd}
\end{equation}
The surge forces from the propellers are calculated as the propeller thrust multiplied by a thrust deduction coefficient $X_{Tport}=X_{Tstbd}=(1-t_{df})$, as in \autoref{eq:X_P_port} and a small yawing moment contribution as in \autoref{eq:N_P_port}, where $y_{pport}$ is the propellers transverse coordinate. The thrusts from the propellers $T_{port}, T_{stbd}$ are taken from the model tests measurements or VCT calculations since modeling of propeller forces is not this paper's focus.
\begin{equation}
    \label{eq:X_P_port}
    X_{P port} = X_{Tport} T_{port}
\end{equation}
\begin{equation}
    \label{eq:N_P_port}
    N_{P port} = - X_{P port} y_{p port}
\end{equation}

%Rudder hull interaction:
An interaction effect exists between the rudder and hull. The flow in the ship's aft is influenced by the rudder, which generates lift on the hull surface. Forces from rudder actions are thus generated on the rudder and the hull. This effect is modeled by the coefficients $\alpha_H$ and $x_H$, as shown in \autoref{eq:Y_RHI} and \autoref{eq:N_RHI}. This is a changed formulation to the MMG model, avoiding coupled coefficients to simplify for regression.
\begin{equation}
    \label{eq:Y_RHI}
    Y_{RHI} = Y_{R} a_{H}
\end{equation}
%
\begin{equation}
    \label{eq:N_RHI}
    N_{RHI} = N_{R} x_{H}
\end{equation}

The mathematical rudder model (\autoref{sec:mathematical_rudder_model}) is expressed as a truncated third-order Taylor expansion, similar to \citet{abkowitz_ship_1964}. 
The semi-empirical rudder is a new compilation of existing semi-empirical formulas from the literature, presented in the next section.
