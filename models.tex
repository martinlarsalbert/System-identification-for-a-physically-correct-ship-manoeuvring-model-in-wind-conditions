\subsection{Models}
\label{sec:models}
The kinematics of the manoeuvring models are described by \autoref{eq:X}-\autoref{eq:N}, where $X_D$, $Y_D$, and $N_D$ describe the total external forces and moments acting on the ship in the surge, sway, and yaw degrees of freedoms.
\begin{equation}
    \label{eq:X}
    m \left(\dot{u} - r^{2} x_{G} - r v\right) = X_{D} + X_{\dot{u}} \dot{u}
\end{equation}
%
\begin{equation}
    \label{eq:Y}
    m \left(\dot{r} x_{G} + \dot{v} + r u\right) = Y_{D} + Y_{\dot{r}} \dot{r} + Y_{\dot{v}} \dot{v}
\end{equation}
%
\begin{equation}
    \label{eq:N}
    I_{z} \dot{r} + m x_{G} \left(\dot{v} + r u\right) = N_{D} + N_{\dot{r}} \dot{r} + N_{\dot{v}} \dot{v}
\end{equation}

The external forces are expressed in a modular way similar to the MMG model \citep{yasukawa_introduction_2015},
% Components:
\begin{equation}
    \label{eq:X_D}
    X_{D} = X_{H} + X_{P} + X_{R}
\end{equation}
%
\begin{equation}
    \label{eq:Y_D}
    Y_{D} = Y_{H} + Y_{P} + Y_{R} + Y_{RHI}
\end{equation}
%
\begin{equation}
    \label{eq:N_D}
    N_{D} = N_{H} + N_{P} + N_{R} + N_{RHI}
\end{equation}
where the subscripts: H, P, R, and RHI represent contributions from: hull, propellers, rudders, and rudder hull interaction.

% Hull:
The hull forces are expressed with the following polynomials expressed in prime system units,
\begin{equation}
    \label{eq:X_H}
    {X_H'} = {X_{0}'} + {X_{rr}'} {r'}^{2} + {X_{u}'} {u'} + {X_{vr}'} {r'} {v'} + {X_{vv}'} {v'}^{2}
\end{equation}
%
\begin{equation}
    \label{eq:Y_H}
    {Y_H'} = {Y_{0}'} + {Y_{r}'} {r'} + {Y_{v}'} {v'} + {\cancel{Y_{rrr}}'} {r'}^{3} + {\cancel{Y_{vrr}}'} {r'}^{2} {v'} + {\cancel{Y_{vvr}}'} {r'} {v'}^{2} + {\cancel{Y_{vvv}}'} {v'}^{3}
\end{equation}
%
\begin{equation}
    \label{eq:N_H}
    {N_H'} = {N_{0}'} + {N_{r}'} {r'} + {N_{v}'} {v'} + {\cancel{N_{rrr}}'} {r'}^{3} + {\cancel{N_{vrr}}'} {r'}^{2} {v'} + {\cancel{N_{vvr}}'} {r'} {v'}^{2} + {\cancel{N_{vvv}}'} {v'}^{3}
\end{equation}
%Propellers:
The propeller forces are expressed as,
\begin{equation}
    \label{eq:X_P}
    X_{P} = X_{P port} + X_{P stbd}
\end{equation}
%
\begin{equation}
    \label{eq:Y_P}
    Y_{P} = 0
\end{equation}
%
\begin{equation}
    \label{eq:N_P}
    N_{P} = N_{P port} + N_{P stbd}
\end{equation}
The surge forces from the propellers are calculated as the propeller thrust times a thrust deduction coefficient $X_{thrust}=(1-t)$ as shown in for the port propeller in \autoref{eq:X_P_port} and a small yawing moment contribution as seen in \autoref{eq:N_P_port}.
\begin{equation}
    \label{eq:X_P_port}
    X_{P port} = X_{Tport} T_{port}
\end{equation}
\begin{equation}
    \label{eq:N_P_port}
    N_{P port} = - X_{P port} y_{p port}
\end{equation}

%Rudder hull interaction:
There is an interaction effect between the rudder and hull. The flow in the ship's aft is influenced by the rudder -- which generates lift on the hull surface -- so that forces from rudder actions are generated both on the rudder and on the hull. This effect is modelled by the coefficients $\alpha_H$ and $x_H$ as seen in \autoref{eq:Y_RHI} and \autoref{eq:N_RHI}.
\begin{equation}
    \label{eq:Y_RHI}
    Y_{RHI} = Y_{R} a_{H}
\end{equation}
%
\begin{equation}
    \label{eq:N_RHI}
    N_{RHI} = N_{R} x_{H}
\end{equation}

In this paper two different rudder models are used for the rudder forces: a pure mathematical rudder model, and a semi-empirical rudder model. The manoeuvring model equipped with the pure mathematical rudder model is referred to as the \emph{``Abkowitz model''}. The other model, equipped with the semi-empirical rudder, is referred to as the \emph{``Semi-empirical model''}. The only things that differ between these models are: the rudder model, and that the Abkowitz model does not have a rudder hull interaction, so that $Y_{RHI}=0$, and $N_{RHI}=0$ for this model.