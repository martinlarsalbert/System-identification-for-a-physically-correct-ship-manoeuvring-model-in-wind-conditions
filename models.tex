\subsection{Models}
\label{sec:models}
The ship's kinematics of the manoeuvring models are described by \autoref{eq:X}-\autoref{eq:N}, where $X_D$, $Y_D$, and $N_D$ describe the total external forces and moments acting on the ship in the surge, sway, and yaw degrees of freedoms.
\begin{equation}
    \label{eq:X}
    m \left(\dot{u} - r^{2} x_{G} - r v\right) = X_{D}
\end{equation}
%
\begin{equation}
    \label{eq:Y}
    m \left(\dot{r} x_{G} + \dot{v} + r u\right) = Y_{D}
\end{equation}
%
\begin{equation}
    \label{eq:N}
    I_{z} \dot{r} + m x_{G} \left(\dot{v} + r u\right) = N_{\dot{r}} \dot{r} + N_{\dot{v}} \dot{v} + X_{D}
\end{equation}

The external forces are expressed in a modular way, similar to the MMG model \citep{yasukawa_introduction_2015},
% Components:
\begin{equation}
    \label{eq:X_D}
    \input{equations/mathematical_model_kinetics.X_D}
\end{equation}
%
\begin{equation}
    \label{eq:Y_D}
    \input{equations/mathematical_model_kinetics.Y_D}
\end{equation}
%
\begin{equation}
    \label{eq:N_D}
    \input{equations/mathematical_model_kinetics.N_D}
\end{equation}
where the subscripts: H, P, R, and RHI represent contributions from: hull, propellers, rudders, and rudder hull interaction.

% Hull:
The hull forces are expressed with the following polynomials, which are here expressed in prime system units,
\begin{equation}
    \label{eq:X_H}
    \input{equations/mathematical_model_kinetics.X_H}
\end{equation}
%
\begin{equation}
    \label{eq:Y_H}
    \input{equations/mathematical_model_kinetics.Y_H}
\end{equation}
%
\begin{equation}
    \label{eq:N_H}
    \input{equations/mathematical_model_kinetics.N_H}
\end{equation}
%Propellers:
The total twin screw propeller forces are expressed as,
\begin{equation}
    \label{eq:X_P}
    \input{equations/mathematical_model_kinetics.X_P}
\end{equation}
%
\begin{equation}
    \label{eq:Y_P}
    \input{equations/mathematical_model_kinetics.Y_P}
\end{equation}
%
\begin{equation}
    \label{eq:N_P}
    \input{equations/mathematical_model_kinetics.N_P}
\end{equation}
The surge forces from the propellers are calculated as the propeller thrust times a thrust deduction coefficient $X_{thrust}=(1-t)$ as shown in \autoref{eq:X_P_port} and a small yawing moment contribution as seen in \autoref{eq:N_P_port}, where $y_{pport}$ is the propellers transverse coordinate.
\begin{equation}
    \label{eq:X_P_port}
    X_{P } = X_{T} T_{}
\end{equation}
\begin{equation}
    \label{eq:N_P_port}
    N_{P } = - X_{P } y_{p }
\end{equation}

%Rudder hull interaction:
There is an interaction effect between the rudder and hull; the flow in the ship's aft is influenced by the rudder -- which generates lift on the hull surface -- so that forces from rudder actions are generated both on the rudder and on the hull. This effect is modelled by the coefficients $\alpha_H$ and $x_H$ as seen in \autoref{eq:Y_RHI} and \autoref{eq:N_RHI}.
\begin{equation}
    \label{eq:Y_RHI}
    \input{equations/mathematical_model_kinetics.Y_RHI}
\end{equation}
%
\begin{equation}
    \label{eq:N_RHI}
    \input{equations/mathematical_model_kinetics.N_RHI}
\end{equation}

In this paper two different rudder models are used for the rudder forces: a pure mathematical rudder model, and a semi-empirical rudder model. The manoeuvring model equipped with the pure mathematical rudder model is referred to as the \emph{``Abkowitz model''}. The other model, equipped with the semi-empirical rudder, is referred to as the \emph{``Semi-empirical model''}. The only things that differ between these models are: the rudder model, and that the Abkowitz model does not have a rudder hull interaction, so that $Y_{RHI}=0$, and $N_{RHI}=0$ for this model.