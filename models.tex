The ship's kinematics of the manoeuvring models are described by \autoref{eq:X}-\autoref{eq:N}, where $X_D$, $Y_D$, and $N_D$ describe the damping forces and moments acting on the ship in the surge, sway, and yaw degrees of freedoms. These equations have the additional added mass terms: $Y_{\dot{r}}$, and $N_{\dot{v}}$, compared to the MMG model.

\begin{equation}
    \label{eq:X}
    m \left(\dot{u} - r^{2} x_{G} - r v\right) = X_{D} + X_{\dot{u}} \dot{u}
\end{equation}
%
\begin{equation}
    \label{eq:Y}
    m \left(\dot{r} x_{G} + \dot{v} + r u\right) = Y_{D} + Y_{\dot{r}} \dot{r} + Y_{\dot{v}} \dot{v}
\end{equation}
%
\begin{equation}
    \label{eq:N}
    I_{z} \dot{r} + m x_{G} \left(\dot{v} + r u\right) = N_{D} + N_{\dot{r}} \dot{r} + N_{\dot{v}} \dot{v}
\end{equation}

The damping forces and moment are expressed in a modular way as seen in \autoref{eq:X_D} to \autoref{eq:N_D} and \autoref{fig:force_model},
% Components:
\begin{equation}
    \label{eq:X_D}
    X_{D} = X_{H} + X_{P} + X_{R}
\end{equation}
%
\begin{equation}
    \label{eq:Y_D}
    Y_{D} = Y_{H} + Y_{P} + Y_{R} + Y_{RHI}
\end{equation}
%
\begin{equation}
    \label{eq:N_D}
    N_{D} = N_{H} + N_{P} + N_{R} + N_{RHI}
\end{equation}
%
\begin{figure}[h]
    \centering
    \includesvg[width=4cm]{figures/force_model.svg}
    \caption{The modular force components.}
    \label{fig:force_model}
\end{figure}
%
where the subscripts: H, P, R, and RHI represent contributions from: hull, propellers, rudders, and rudder hull interaction. Rudder hull interaction having its own element is a difference to the MMG model, where RHI is included in the hull forces.
% Hull:

The hull forces are expressed with the same polynomials as the MMG model, except that the ${X_{vvvv}}'$ coefficient has been omitted and an additional resistance term ${X_u}'$ has been added -- to allow for a more nonlinear resistance (see \autoref{sec:hull}). The nonlinear resistance is possible due to the use of perturbed velocity (see \autoref{sec:prime_system}).

%Propellers:
The total twin screw propeller forces are expressed as,
\begin{equation}
    \label{eq:X_P}
    X_{P} = X_{P port} + X_{P stbd}
\end{equation}
%
\begin{equation}
    \label{eq:Y_P}
    Y_{P} = 0
\end{equation}
%
\begin{equation}
    \label{eq:N_P}
    N_{P} = N_{P port} + N_{P stbd}
\end{equation}
The surge forces from the propellers are calculated as the propeller thrust times a thrust deduction coefficient $X_{thrust}=(1-t)$ as shown in \autoref{eq:X_P_port} and a small yawing moment contribution as seen in \autoref{eq:N_P_port}, where $y_{pport}$ is the propellers transverse coordinate.
\begin{equation}
    \label{eq:X_P_port}
    X_{P port} = X_{Tport} T_{port}
\end{equation}
\begin{equation}
    \label{eq:N_P_port}
    N_{P port} = - X_{P port} y_{p port}
\end{equation}

%Rudder hull interaction:
There is an interaction effect between the rudder and hull; the flow in the ship's aft is influenced by the rudder -- which generates lift on the hull surface -- so that forces from rudder actions are generated both on the rudder and on the hull. This effect is modelled by the coefficients $\alpha_H$ and $x_H$ as seen in \autoref{eq:Y_RHI} and \autoref{eq:N_RHI}. This is a changed formulation to the MMG model, to simplify for regression.
\begin{equation}
    \label{eq:Y_RHI}
    Y_{RHI} = Y_{R} a_{H}
\end{equation}
%
\begin{equation}
    \label{eq:N_RHI}
    N_{RHI} = N_{R} x_{H}
\end{equation}

The mathematical rudder model (\autoref{sec:mathematical_rudder_model}) is expressed as a truncated third order Taylor expansion, similar to \citet{abkowitz_ship_1964}. 

The semi-empirical rudder (\autoref{sec:semiempirical_rudder_model}) is a new compilation of existing semi-empirical formulas from the literature.
