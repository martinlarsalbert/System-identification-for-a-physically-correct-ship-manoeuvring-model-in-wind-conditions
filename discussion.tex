The following points should be considered:
\begin{itemize}
    \item Having accurate values for the added masses is very important for the ID to deliver forces of the correct magnitude, especially when a deterministic model, such as the semi-empirical rudder, is combined with data-driven models, such as the hull model in this paper. 
    \item The semi-empirical rudder was actually not treated as an entirely deterministic model in this paper. The flow straightening coefficients $\kappa_v$ and $\kappa_r$ were determined from the VCT as well as the rudder hull interaction coefficients $a_H$ and $x_H$. For the time being, some VCTs are, therefore, needed for this model. In the future, when more experience is gained about these coefficients, semi-empirical expressions or rules of thumb can hopefully be developed to make the model fully deterministic.
    \item In the persistence of excitation, the zigzag test is not very informative \citep{sutulo_algorithm_2014}.
    \item For instances of pseudo-random binary sequences (PRBSs), \citep{landau_digital_2006} is more informative.
\end{itemize}