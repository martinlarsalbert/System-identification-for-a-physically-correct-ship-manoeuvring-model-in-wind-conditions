The lift coefficient $C_L$ is calculated with \autoref{eq:C_L_semiempirical}, which contains two separate expressions for angle of attack $\alpha$ (\autoref{eq:alpha_semiempirical}) below or above a critical stalling angle $\alpha_s$ (\autoref{eq:alpha_s_semiempirical}).
\begin{equation}
    \label{eq:C_L_semiempirical}
    C_{L } = \lambda_{gap } \left(\alpha_{} dC_{L dalpha } + \frac{C_{D crossflow } \alpha_{} \left|{\alpha_{}}\right|}{AR_{e }}\right)
\end{equation}
%
\begin{equation}
    \label{eq:alpha_s_semiempirical}
    \input{equations/mathematical_model_kinetics.alpha_s_semiempirical}
\end{equation}
%
\begin{equation}
    \label{eq:alpha_semiempirical}
    \input{equations/mathematical_model_kinetics.alpha_semiempirical}
\end{equation}
Before stall, the lift slope of the rudder $\frac{\partial C_L}{\partial \alpha}$ is calculated with \autoref{eq:dC_L_dalpha_semiempirical} where the effective aspect ratio of the rudder $AR_e$ can be obtained from \autoref{eq:AR_e_semiempirical}. $AR_e$ is the geometrical aspect ratio $AR_g$ (\autoref{eq:AR_g_semiempirical}) with a correction for a possible gap between the upper part of the rudder and the rudder horn for larger rudder angles.
$a_0$ is the section lift curve slope (\autoref{eq:a_0_semiempirical}) and $\Omega$ is the sweep angle of the quarter chord line \citep{lewis_principles_1989}.
\begin{equation}
    \label{eq:dC_L_dalpha_semiempirical}
    dC_{L dalpha } = \frac{AR_{e } a_{0 }}{\sqrt{\frac{AR_{e }^{2}}{\cos^{4}{\left(\Omega \right)}} + 4} \cos{\left(\Omega \right)} + 1.8}
\end{equation}
%
\begin{equation}
    \label{eq:AR_e_semiempirical}
    AR_{e } = AR_{g } \left(2 - \left|{\frac{\\delta}{\\delta_{lim}}}\right|\right)
\end{equation}
%
\begin{equation}
    \label{eq:AR_g_semiempirical}
    \input{equations/mathematical_model_kinetics.AR_g_semiempirical}
\end{equation}
%
\begin{equation}
    \label{eq:a_0_semiempirical}
    \input{equations/mathematical_model_kinetics.a_0_semiempirical}
\end{equation}
There is also a small nonlinear part to $C_L$ called $C_{Dcrossflow}$ which is calculated for a rudder with squared tip using \autoref{eq:C_D_crossflow_semiempirical} where the taper ratio $\lambda$ is defined as the ratio between the chords at the tip and the root of the rudder (\autoref{eq:lambda__semiempirical}) \citep{hughes_tempest_2011}. 
\begin{equation}
    \label{eq:C_D_crossflow_semiempirical}
    C_{D crossflow } = 1.6 \lambda^{} + 0.1
\end{equation}
%
\begin{equation}
    \label{eq:lambda__semiempirical}
    \input{equations/mathematical_model_kinetics.lambda__semiempirical}
\end{equation}

Expressions to calculate: $C_{Lmax}$, $B_0$, $B_s$, $u_s$, and $C_N$; to get the lift after stall are shown in \autoref{eq:C_L_max_semiempirical} to \autoref{eq:C_N_semiempirical}.
\begin{equation}
    \label{eq:C_L_max_semiempirical}
    C_{L max } = \alpha_{s } dC_{L dalpha } + \frac{C_{D crossflow } \alpha_{s } \left|{\alpha_{s }}\right|}{AR_{e }}
\end{equation}
%
\begin{equation}
    \label{eq:B_0_semiempirical}
    \input{equations/mathematical_model_kinetics.B_0_semiempirical}
\end{equation}
%
\begin{equation}
    \label{eq:B_s_semiempirical}
    \input{equations/mathematical_model_kinetics.B_s_semiempirical}
\end{equation}
%
\begin{equation}
    \label{eq:u_s_semiempirical}
    \input{equations/mathematical_model_kinetics.u_s_semiempirical}
\end{equation}
%
\begin{equation}
    \label{eq:C_N_semiempirical}
    \input{equations/mathematical_model_kinetics.C_N_semiempirical}
\end{equation}