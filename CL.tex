For a none stalling rudder, the lift coefficient $C_L$ is calculated with \autoref{eq:C_L_semiempirical},
\begin{equation}
    \label{eq:C_L_semiempirical}
    C_{L} = \lambda_{gap} \left(\alpha_} dC_{L dalpha} + \frac{C_{DC} \alpha_} \left|{\alpha_}}\right|}{AR_{e}}\right)
\end{equation}
%
\begin{equation}
    \label{eq:alpha_semiempirical}
    \alpha_{} = \\delta + \gamma_{0 } + \gamma_{}
\end{equation}
The lift slope of the rudder $\frac{\partial C_L}{\partial \alpha}$ is calculated with \autoref{eq:dC_L_dalpha_semiempirical} where the effective aspect ratio of the rudder $AR_e$ can be obtained from \autoref{eq:AR_e_semiempirical}. $AR_e$ is the geometrical aspect ratio $AR_g$ (\autoref{eq:AR_g_semiempirical}) with a correction for a possible gap between the upper part of the rudder and the rudder horn for larger rudder angles.
$a_0$ is the section lift curve slope (\autoref{eq:a_0_semiempirical}) and $\Omega$ is the sweep angle of the quarter chord line \citep{lewis_principles_1989}.
\begin{equation}
    \label{eq:dC_L_dalpha_semiempirical}
    dC_{L dalpha} = \frac{AR_{e} a_{0}}{\sqrt{\frac{AR_{e}^{2}}{\cos^{4}{\left(\Omega \right)}} + 4} \cos{\left(\Omega \right)} + 1.8}
\end{equation}
%
\begin{equation}
    \label{eq:AR_e_semiempirical}
    AR_{e} = 2 AR_{g}
\end{equation}
%
\begin{equation}
    \label{eq:AR_g_semiempirical}
    AR_{g} = \frac{b_{R}^{2}}{A_{R}}
\end{equation}
%
\begin{equation}
    \label{eq:a_0_semiempirical}
    a_{0 } = 1.8 \pi
\end{equation}
There is also a small nonlinear part to $C_L$ called $C_{Dcrossflow}$ which is calculated for a rudder with squared tip using \autoref{eq:C_D_crossflow_semiempirical} where the taper ratio $\lambda$ is defined as the ratio between the chords at the tip and the root of the rudder (\autoref{eq:lambda__semiempirical}) \citep{hughes_tempest_2011}. 
\begin{equation}
    \label{eq:C_D_crossflow_semiempirical}
    C_{DC} = 1.6 \lambda^{} + 0.1
\end{equation}
%
\begin{equation}
    \label{eq:lambda__semiempirical}
    \lambda} = \frac{c_{t}}{c_{r}}
\end{equation}

Expressions to calculate: $C_{Lmax}$, $B_0$, $B_s$, $u_s$, and $C_N$; to get the lift after stall are shown in \autoref{eq:C_L_max_semiempirical} to \autoref{eq:C_N_semiempirical}.
\begin{equation}
    \label{eq:C_L_max_semiempirical}
    C_{L max } = \alpha_{s } \frac{\partial C_L}{\partial \alpha + \frac{C_{D crossflow } \alpha_{s } \left|{\alpha_{s }}\right|}{AR_{e }}
\end{equation}
%
\begin{equation}
    \label{eq:B_0_semiempirical}
    B_{0} = 1 - B_{s}
\end{equation}
%
\begin{equation}
    \label{eq:B_s_semiempirical}
    B_{s} = \begin{cases} 1 & \text{for}\: \left|{\alpha_}}\right| > 1.25 \alpha_{s} \\u_{s} \left(- 2 u_{s}^{2} + 3 u_{s}\right) & \text{otherwise} \end{cases}
\end{equation}
%
\begin{equation}
    \label{eq:u_s_semiempirical}
    u_{s } = \frac{- 4 \alpha_{s } + 4 \left|{\alpha_{}}\right|}{\alpha_{s }}
\end{equation}
%
\begin{equation}
    \label{eq:C_N_semiempirical}
    C_{N } = 1.8
\end{equation}