According to momentum theory the mean axial flow velocity far downstream of the propeller $V_{\infty}$ is given by \autoref{eq:V_infty_semiempirical} \cite{brix_manoeuvring_1993} where the thrust coefficient $C_{Th}$ is calculated with \autoref{eq:C_Th_semiempirical} where $r_0$ is the propeller radius and the apparent velocity $V_A$ is given by \autoref{eq:V_A_semiempirical}.
\begin{equation}
    \label{eq:V_infty_semiempirical}
    V_{\infty } = V_{A } \sqrt{C_{Th } + 1}
\end{equation}
%
\begin{equation}
    \label{eq:C_Th_semiempirical}
    C_{Th } = \frac{2 T_{}}{\pi V_{A }^{2} r_{0}^{2} \rho}
\end{equation}
%
\begin{equation}
    \label{eq:V_A_semiempirical}
    V_{A} = u \left(1 - w_{f}\right)
\end{equation}

The radius of the propeller slipstream far behind the propeller is given by \autoref{eq:r_infty_semiempirical}.
\begin{equation}
    \label{eq:r_infty_semiempirical}
    r_{\infty } = r_{0} \sqrt{\frac{V_{A }}{2 V_{\infty }} + \frac{1}{2}}
\end{equation}
The velocity and the radius of the propeller slipstream at the position of the rudder can be calculated with \autoref{eq:V_x_C_semiempirical} and \autoref{eq:r_p_semiempirical} where $x$ is the distance between the propeller and the rudder.
\begin{equation}
    \label{eq:V_x_C_semiempirical}
    V_{x C } = \frac{V_{\infty } r_{\infty }^{2}}{r_{x }^{2}}
\end{equation}
%
\begin{equation}
    \label{eq:r_p_semiempirical}
    r_{x } = \frac{r_{0} \left(\frac{r_{\infty } \left(\frac{x}{r_{0}}\right)^{1.5}}{r_{0}} + \frac{0.14 r_{\infty }^{3}}{r_{0}^{3}}\right)}{\left(\frac{x}{r_{0}}\right)^{1.5} + \frac{0.14 r_{\infty }^{3}}{r_{0}^{3}}}
\end{equation}
Turbulent mixing of the slipstream and the surrounding flow will increase the radius $r_x$ by $r_\Delta$(\autoref{eq:r_Delta_semiempirical}) so that a corrected axial velocity $V_{xcorr}$ can be calculated according to \autoref{eq:V_x_corr_semiempirical}.
\begin{equation}
    \label{eq:r_Delta_semiempirical}
    r_{\Delta } = \frac{0.15 x \left(- V_{A } + V_{x C }\right)}{V_{A } + V_{x C }}
\end{equation}
%
\begin{equation}
    \label{eq:V_x_corr_semiempirical}
    V_{x corr } = V_{A } + \frac{r_{x }^{2} \left(- V_{A } + V_{x C }\right)}{\left(r_{\Delta } + r_{x }\right)^{2}}
\end{equation}
For a twin screw ship a small contribution from the yaw rate is also added to the velocity as seen in \autoref{eq:V_R_x_C_semiempirical}.
\begin{equation}
    \label{eq:V_R_x_C_semiempirical}
    V_{R x C} = V_{x corr} - r y_{R}
\end{equation}
The velocity for the covered part of the rudder is obtained by \autoref{eq:V_R_C_semiempirical}.
\begin{equation}
    \label{eq:V_R_C_semiempirical}
    V_{R C} = \sqrt{V_{R x C}^{2} + V_{R y}^{2}}
\end{equation}
$V_{xcorr}$ is also used to calculate the lift diminished factor $\lambda_R$ together with the expressions in  \autoref{eq:lambda_R_semiempirical} to \autoref{eq:c_semiempirical}.
\begin{equation}
    \label{eq:lambda_R_semiempirical}
    \lambda_{R } = \left(\frac{V_{A }}{V_{x corr }}\right)^{f_{}}
\end{equation}
%
\begin{equation}
    \label{eq:f_semiempirical}
    f_} = \frac{512}{\left(2 + \frac{d_}}{c_}}\right)^{8}}
\end{equation}
%
\begin{equation}
    \label{eq:d_semiempirical}
    d_{} = \frac{\sqrt{\pi} \left(r_{\Delta } + r_{p }\right)}{2}
\end{equation}
\begin{equation}
    \label{eq:c_semiempirical}
    c_} = \frac{c_{r}}{2} + \frac{c_{t}}{2}
\end{equation}