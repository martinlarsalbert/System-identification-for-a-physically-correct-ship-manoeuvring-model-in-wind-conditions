\subsection{Semi-empirical rudder model}
\label{sec:semiempirical_rudder_model}
The semi-empirical rudder model is a lifting line model. The surge and sway force are expressed as rudder lift $L_R$, and drag $D_R$ projected on the ship through the rudder inflow angle $\alpha_f$ as seen in \autoref{eq:X_R_semiempirical} and \autoref{eq:Y_R_semiempirical}.
This angle is a sum of the initial inflow to the rudder at a straight course $\gamma_0$ and the inflow to the rudder $\gamma$ due to propeller induced speed, drift angle, and yaw rate of the ship as seen in \autoref{eq:gamma_semiempirical}.
\begin{equation}
    \label{eq:X_R_semiempirical}
    \input{equations/mathematical_model_kinetics.X_R_semiempirical}
\end{equation}
%
\begin{equation}
    \label{eq:Y_R_semiempirical}
    \input{equations/mathematical_model_kinetics.Y_R_semiempirical}
\end{equation}
%
\begin{equation}
    \label{eq:alpha_f_semiempirical}
    \input{equations/mathematical_model_kinetics.alpha_f_semiempirical}
\end{equation}
%
\begin{equation}
    \label{eq:gamma_semiempirical}
    \input{equations/mathematical_model_kinetics.gamma_semiempirical}
\end{equation}
The transverse velocity at the rudder $V_{Ry}$ is calculated with the ship's yaw rate $r$ and the transverse velocity $v$ multiplied by their flow straightening coefficients $\kappa_{rtot}$ and $\kappa_{vtot}$ (\autoref{eq:V_R_y_semiempirical}). The longitudinal velocity at the rudder $V_{RxC}$ is calculated with \autoref{eq:V_R_x_C_semiempirical}.
\begin{equation}
    \label{eq:V_R_y_semiempirical}
    \input{equations/mathematical_model_kinetics.V_R_y_semiempirical}
\end{equation}

The yaw moment is modelled as the sway force times a lever arm $x_R$ as seen in \autoref{eq:N_R_semiempirical}.
\begin{equation}
    \label{eq:N_R_semiempirical}
    \input{equations/mathematical_model_kinetics.N_R_semiempirical}
\end{equation}

\subsubsection{Rudder lift}
\label{sec:rudder lift}
The total rudder lift is calculated as the sum of the lift from the part of the rudder that is covered by the propeller race $L_{RC}$ and the uncovered part $L_{RU}$ as seen in \autoref{eq:L_R_semiempirical}.
\begin{equation}
    \label{eq:L_R_semiempirical}
    \input{equations/mathematical_model_kinetics.L_R_semiempirical}
\end{equation}
The lift forces are calculated with the lift coefficient $C_L$ as seen in \autoref{eq:L_R_U_semiempirical} and \autoref{eq:L_R_C_semiempirical}. These equations are essentially the same except that the lift by the covered part $L_{RC}$ is diminished by the factor $\lambda_R$ (\autoref{eq:lambda_R_semiempirical}) due to the limited radius of the propeller slipstream in the lateral direction \citep{brix_manoeuvring_1993}. A tuning coefficient $C_{Ltune}$ has also been added to these equations.
\begin{equation}
    \label{eq:L_R_U_semiempirical}
    \input{equations/mathematical_model_kinetics.L_R_U_semiempirical}
\end{equation}
%
\begin{equation}
    \label{eq:L_R_C_semiempirical}
    \input{equations/mathematical_model_kinetics.L_R_C_semiempirical}
\end{equation}

The lift coefficient $C_L$ is calculated with \autoref{eq:C_L_semiempirical}, which contains two separate expressions for angle of attack $\alpha$ below or above a critical stalling angle $\alpha_s$.
\begin{equation}
    \label{eq:C_L_semiempirical}
    C_{L } = \lambda_{gap } \left(\alpha_{} dC_{L dalpha } + \frac{C_{D crossflow } \alpha_{} \left|{\alpha_{}}\right|}{AR_{e }}\right)
\end{equation}
The lift slope of the rudder $\frac{\partial C_L}{\partial \alpha}$ is calculated with \autoref{eq:dC_L_dalpha_semiempirical} where the effective aspect ratio of the rudder $AR_e$ can be obtained from \autoref{eq:AR_e_semiempirical}. $AR_e$ is the geometrical aspect ratio $AR_g$ (\autoref{eq:AR_g_semiempirical}) with a correction for a potential gap between the upper part of the rudder and the rudder horn for larger rudder angles.
$a_0$ is the section lift curve slope (\autoref{eq:a_0_semiempirical}) and $\Omega$ is the sweep angle of the quarter chord line.
\begin{equation}
    \label{eq:dC_L_dalpha_semiempirical}
    dC_{L dalpha } = \frac{AR_{e } a_{0 }}{\sqrt{\frac{AR_{e }^{2}}{\cos^{4}{\left(\Omega \right)}} + 4} \cos{\left(\Omega \right)} + 1.8}
\end{equation}
%
\begin{equation}
    \label{eq:AR_e_semiempirical}
    AR_{e } = AR_{g } \left(2 - \left|{\frac{\\delta}{\\delta_{lim}}}\right|\right)
\end{equation}
%
\begin{equation}
    \label{eq:AR_g_semiempirical}
    \input{equations/mathematical_model_kinetics.AR_g_semiempirical}
\end{equation}
%
\begin{equation}
    \label{eq:a_0_semiempirical}
    \input{equations/mathematical_model_kinetics.a_0_semiempirical}
\end{equation}
There is also a small nonlinear part to $C_L$ called $C_{Dcrossflow}$ which is calculated for a rudder with squared tip using \autoref{eq:C_D_crossflow_semiempirical} where the taper ratio $\lambda$ is defined as the ratio between the chords at the tip and the root of the rudder \citep{hughes_tempest_2011}. 
\begin{equation}
    \label{eq:C_D_crossflow_semiempirical}
    C_{D crossflow } = 1.6 \lambda^{} + 0.1
\end{equation}
%
\begin{equation}
    \label{eq:lambda__semiempirical}
    \input{equations/mathematical_model_kinetics.lambda__semiempirical}
\end{equation}
%
\begin{equation}
    \label{eq:lambda_R_semiempirical}
    \input{equations/mathematical_model_kinetics.lambda_R_semiempirical}
\end{equation}
%
\begin{equation}
    \label{eq:f_semiempirical}
    \input{equations/mathematical_model_kinetics.f_semiempirical}
\end{equation}
%
\begin{equation}
    \label{eq:d_semiempirical}
    d_{} = \frac{\sqrt{\pi} \left(r_{\Delta } + r_{p }\right)}{2}
\end{equation}
%
\begin{equation}
    \label{eq:V_x_corr_semiempirical}
    V_{x corr } = V_{A } + \frac{r_{p }^{2} \left(- V_{A } + V_{x C }\right)}{\left(r_{\Delta } + r_{p }\right)^{2}}
\end{equation}
%
\begin{equation}
    \label{eq:r_Delta_semiempirical}
    \input{equations/mathematical_model_kinetics.r_Delta_semiempirical}
\end{equation}
%
\begin{equation}
    \label{eq:D_R_semiempirical}
    \input{equations/mathematical_model_kinetics.D_R_semiempirical}
\end{equation}
%
\begin{equation}
    \label{eq:C_D_C_semiempirical}
    \input{equations/mathematical_model_kinetics.C_D_C_semiempirical}
\end{equation}
%
\begin{equation}
    \label{eq:C_D0_C_semiempirical}
    \input{equations/mathematical_model_kinetics.C_D0_C_semiempirical}
\end{equation}
%
\begin{equation}
    \label{eq:C_F_C_semiempirical}
    \input{equations/mathematical_model_kinetics.C_F_C_semiempirical}
\end{equation}
%
\begin{equation}
    \label{eq:Re_F_C_semiempirical}
    \input{equations/mathematical_model_kinetics.Re_F_C_semiempirical}
\end{equation}
%
\begin{equation}
    \label{eq:V_R_C_semiempirical}
    \input{equations/mathematical_model_kinetics.V_R_C_semiempirical}
\end{equation}
%
\begin{equation}
    \label{eq:C_D_U_semiempirical}
    \input{equations/mathematical_model_kinetics.C_D_U_semiempirical}
\end{equation}
%
\begin{equation}
    \label{eq:C_N_semiempirical}
    \input{equations/mathematical_model_kinetics.C_N_semiempirical}
\end{equation}
%
\begin{equation}
    \label{eq:C_L_max_semiempirical}
    C_{L max } = \alpha_{s } dC_{L dalpha } + \frac{C_{D crossflow } \alpha_{s } \left|{\alpha_{s }}\right|}{AR_{e }}
\end{equation}
%
\begin{equation}
    \label{eq:B_0_semiempirical}
    \input{equations/mathematical_model_kinetics.B_0_semiempirical}
\end{equation}
%
\begin{equation}
    \label{eq:B_s_semiempirical}
    \input{equations/mathematical_model_kinetics.B_s_semiempirical}
\end{equation}
%
\begin{equation}
    \label{eq:u_s_semiempirical}
    \input{equations/mathematical_model_kinetics.u_s_semiempirical}
\end{equation}
%
\begin{equation}
    \label{eq:alpha_s_semiempirical}
    \input{equations/mathematical_model_kinetics.alpha_s_semiempirical}
\end{equation}
%
\begin{equation}
    \label{eq:alpha_semiempirical}
    \input{equations/mathematical_model_kinetics.alpha_semiempirical}
\end{equation}
%
\begin{equation}
    \label{eq:C_D0_U_semiempirical}
    \input{equations/mathematical_model_kinetics.C_D0_U_semiempirical}
\end{equation}
%
\begin{equation}
    \label{eq:C_F_U_semiempirical}
    \input{equations/mathematical_model_kinetics.C_F_U_semiempirical}
\end{equation}
%
\begin{equation}
    \label{eq:Re_F_U_semiempirical}
    \input{equations/mathematical_model_kinetics.Re_F_U_semiempirical}
\end{equation}
%
\begin{equation}
    \label{eq:V_R_U_semiempirical}
    \input{equations/mathematical_model_kinetics.V_R_U_semiempirical}
\end{equation}
%
\begin{equation}
    \label{eq:V_R_x_U_semiempirical}
    \input{equations/mathematical_model_kinetics.V_R_x_U_semiempirical}
\end{equation}
%
\begin{equation}
    \label{eq:V_x_U_semiempirical}
    \input{equations/mathematical_model_kinetics.V_x_U_semiempirical}
\end{equation}
%
\begin{equation}
    \label{eq:kappa_r_tot_semiempirical}
    \input{equations/mathematical_model_kinetics.kappa_r_tot_semiempirical}
\end{equation}
%
\begin{equation}
    \label{eq:kappa_v_tot_semiempirical}
    \input{equations/mathematical_model_kinetics.kappa_v_tot_semiempirical}
\end{equation}
%
\begin{equation}
    \label{eq:gamma_g_semiempirical}
    \input{equations/mathematical_model_kinetics.gamma_g_semiempirical}
\end{equation}
%
\begin{equation}
    \label{eq:V_R_x_C_semiempirical}
    V_{R x C } = V_{x C } - r y_{R}
\end{equation}
%
\begin{equation}
    \label{eq:V_x_C_semiempirical}
    V_{x C } = \frac{V_{\infty } r_{\infty }^{2}}{r_{p }^{2}}
\end{equation}
%
\begin{equation}
    \label{eq:r_p_semiempirical}
    \input{equations/mathematical_model_kinetics.r_p_semiempirical}
\end{equation}
%
\begin{equation}
    \label{eq:r_infty_semiempirical}
    \input{equations/mathematical_model_kinetics.r_infty_semiempirical}
\end{equation}
%
\begin{equation}
    \label{eq:V_infty_semiempirical}
    \input{equations/mathematical_model_kinetics.V_infty_semiempirical}
\end{equation}
%
\begin{equation}
    \label{eq:C_Th_semiempirical}
    C_{Th} = \frac{2T_}}{\pi V_{A}^{2} r_{0}^{2} \rho}
\end{equation}
%
\begin{equation}
    \label{eq:V_A_semiempirical}
    \input{equations/mathematical_model_kinetics.V_A_semiempirical}
\end{equation}
%
\begin{equation}
    \label{eq:c_semiempirical}
    \input{equations/mathematical_model_kinetics.c_semiempirical}
\end{equation}
%
