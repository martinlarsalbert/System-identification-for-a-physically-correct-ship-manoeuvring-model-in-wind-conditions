The semi-empirical rudder model is a lifting line model. The surge and sway force are expressed as: rudder lift $L_R$, and rudder drag $D_R$; projected on the ship through the rudder inflow angle $\alpha_f$, as seen in \autoref{eq:X_R_semiempirical} and \autoref{eq:Y_R_semiempirical}.
This angle is a sum of the initial inflow to the rudder at a straight course $\gamma_0$ and the inflow to the rudder $\gamma$ due to: propeller induced speed, drift angle, and yaw rate of the ship; as seen in \autoref{eq:gamma_semiempirical}.
\begin{equation}
    \label{eq:X_R_semiempirical}
    X_{R } = - D_{R } \cos{\left(\alpha_{f } \right)} + L_{R } \sin{\left(\alpha_{f } \right)}
\end{equation}
%
\begin{equation}
    \label{eq:Y_R_semiempirical}
    Y_{R} = D_{R} \sin{\left(\alpha_{f} \right)} + L_{R} \cos{\left(\alpha_{f} \right)}
\end{equation}
%
\begin{equation}
    \label{eq:alpha_f_semiempirical}
    \alpha_{f } = \gamma_{0 } + \gamma_{}
\end{equation}
%
\begin{equation}
    \label{eq:gamma_semiempirical}
    \gamma_} = \operatorname{atan}{\left(\frac{V_{R y}}{V_{R x C}} \right)}
\end{equation}
The transverse velocity at the rudder $V_{Ry}$ is calculated with the ship's yaw rate $r$ and the transverse velocity $v$ multiplied by their flow straightening $\kappa_{rtot}$ and $\kappa_{vtot}$ (\autoref{eq:V_R_y_semiempirical}). The flow straightening has a nonlinear dependency of the geometric inflow angle $\gamma_g$ (\autoref{eq:gamma_g_semiempirical}) as seen in \autoref{eq:kappa_r_tot_semiempirical} and \autoref{eq:kappa_v_tot_semiempirical}, so that the flow straightening may vary for different inflow angles.
The axial velocity at the rudder $V_{RxC}$, including the velocity of the propeller race is presented in \autoref{sec:propeller_induced_velocity} with \autoref{eq:V_R_x_C_semiempirical}.
\begin{equation}
    \label{eq:V_R_y_semiempirical}
    V_{R y} = - \kappa_{r tot} r x_{R} - \kappa_{v tot} v
\end{equation}
%
\begin{equation}
    \label{eq:kappa_r_tot_semiempirical}
    \kappa_{r tot} = \kappa_{r} + \kappa_{r \gamma g} \left|{\gamma_{g}}\right|
\end{equation}
%
\begin{equation}
    \label{eq:kappa_v_tot_semiempirical}
    \kappa_{v tot} = \kappa_{v} + \kappa_{v \gamma g} \left|{\gamma_{g}}\right|
\end{equation}
%
\begin{equation}
    \label{eq:gamma_g_semiempirical}
    \gamma_{g} = \operatorname{atan}{\left(\frac{- r x_{R} - v}{V_{R x C}} \right)}
\end{equation}

The yaw moment is modelled as the sway force times a lever arm $x_R$ as seen in \autoref{eq:N_R_semiempirical}.
\begin{equation}
    \label{eq:N_R_semiempirical}
    N_{R} = Y_{R} x_{R}
\end{equation}
%
%
\subsubsection{Rudder lift}
\label{sec:rudder lift}
The total rudder lift is calculated as the sum of the lift from the part of the rudder that is covered by the propeller race $L_{RC}$ and the uncovered part $L_{RU}$ as seen in \autoref{eq:L_R_semiempirical}.
\begin{equation}
    \label{eq:L_R_semiempirical}
    L_{R } = L_{R C } + L_{R U }
\end{equation}
The lift forces are calculated with the lift coefficient $C_L$ as seen in \autoref{eq:L_R_U_semiempirical} and \autoref{eq:L_R_C_semiempirical}. These equations are essentially the same except that the lift by the covered part $L_{RC}$ is diminished by the factor $\lambda_R$ (\autoref{eq:lambda_R_semiempirical}) due to the limited radius of the propeller slipstream in the lateral direction \citep{brix_manoeuvring_1993}. A tuning coefficient $C_{Ltune}$ has also been added to these equations.
\begin{equation}
    \label{eq:L_R_U_semiempirical}
    L_{R U } = \frac{A_{R U} C_{L } C_{L tune} V_{R U }^{2} \rho}{2}
\end{equation}
%
\begin{equation}
    \label{eq:L_R_C_semiempirical}
    L_{R C } = \frac{A_{R C} C_{L } C_{L tune} V_{R C }^{2} \lambda_{R } \rho}{2}
\end{equation}

The lift coefficient $C_L$ is calculated with \autoref{eq:C_L_semiempirical}, which contains two separate expressions for angle of attack $\alpha$ (\autoref{eq:alpha_semiempirical}) below or above a critical stalling angle $\alpha_s$ (\autoref{eq:alpha_s_semiempirical}).
\begin{equation}
    \label{eq:C_L_semiempirical}
    C_{L} = \lambda_{gap} \left(\alpha_} dC_{L dalpha} + \frac{C_{DC} \alpha_} \left|{\alpha_}}\right|}{AR_{e}}\right)
\end{equation}
%
\begin{equation}
    \label{eq:alpha_s_semiempirical}
    \alpha_{s } = \delta_{\alpha s} + \begin{cases} 0.075 AR_{e }^{2} - 0.445 AR_{e } + 1.225 & \text{for}\: AR_{e } \leq 3 \\0.565 & \text{otherwise} \end{cases}
\end{equation}
%
\begin{equation}
    \label{eq:alpha_semiempirical}
    \alpha_{} = \\delta + \gamma_{0 } + \gamma_{}
\end{equation}
Before stall, the lift slope of the rudder $\frac{\partial C_L}{\partial \alpha}$ is calculated with \autoref{eq:dC_L_dalpha_semiempirical} where the effective aspect ratio of the rudder $AR_e$ can be obtained from \autoref{eq:AR_e_semiempirical}. $AR_e$ is the geometrical aspect ratio $AR_g$ (\autoref{eq:AR_g_semiempirical}) with a correction for a possible gap between the upper part of the rudder and the rudder horn for larger rudder angles.
$a_0$ is the section lift curve slope (\autoref{eq:a_0_semiempirical}) and $\Omega$ is the sweep angle of the quarter chord line \citep{lewis_principles_1989}.
\begin{equation}
    \label{eq:dC_L_dalpha_semiempirical}
    dC_{L dalpha} = \frac{AR_{e} a_{0}}{\sqrt{\frac{AR_{e}^{2}}{\cos^{4}{\left(\Omega \right)}} + 4} \cos{\left(\Omega \right)} + 1.8}
\end{equation}
%
\begin{equation}
    \label{eq:AR_e_semiempirical}
    AR_{e} = 2 AR_{g}
\end{equation}
%
\begin{equation}
    \label{eq:AR_g_semiempirical}
    AR_{g} = \frac{b_{R}^{2}}{A_{R}}
\end{equation}
%
\begin{equation}
    \label{eq:a_0_semiempirical}
    a_{0 } = 1.8 \pi
\end{equation}
There is also a small nonlinear part to $C_L$ called $C_{Dcrossflow}$ which is calculated for a rudder with squared tip using \autoref{eq:C_D_crossflow_semiempirical} where the taper ratio $\lambda$ is defined as the ratio between the chords at the tip and the root of the rudder (\autoref{eq:lambda__semiempirical}) \citep{hughes_tempest_2011}. 
\begin{equation}
    \label{eq:C_D_crossflow_semiempirical}
    C_{DC} = 1.6 \lambda^{} + 0.1
\end{equation}
%
\begin{equation}
    \label{eq:lambda__semiempirical}
    \lambda} = \frac{c_{t}}{c_{r}}
\end{equation}

Expressions to calculate: $C_{Lmax}$, $B_0$, $B_s$, $u_s$, and $C_N$; to get the lift after stall are shown in \autoref{eq:C_L_max_semiempirical} to \autoref{eq:C_N_semiempirical}.
\begin{equation}
    \label{eq:C_L_max_semiempirical}
    C_{L max } = \alpha_{s } \frac{\partial C_L}{\partial \alpha + \frac{C_{D crossflow } \alpha_{s } \left|{\alpha_{s }}\right|}{AR_{e }}
\end{equation}
%
\begin{equation}
    \label{eq:B_0_semiempirical}
    B_{0} = 1 - B_{s}
\end{equation}
%
\begin{equation}
    \label{eq:B_s_semiempirical}
    B_{s} = \begin{cases} 1 & \text{for}\: \left|{\alpha_}}\right| > 1.25 \alpha_{s} \\u_{s} \left(- 2 u_{s}^{2} + 3 u_{s}\right) & \text{otherwise} \end{cases}
\end{equation}
%
\begin{equation}
    \label{eq:u_s_semiempirical}
    u_{s } = \frac{- 4 \alpha_{s } + 4 \left|{\alpha_{}}\right|}{\alpha_{s }}
\end{equation}
%
\begin{equation}
    \label{eq:C_N_semiempirical}
    C_{N } = 1.8
\end{equation}
%
%
\subsubsection{Velocity outside the propeller slip stream}
\label{sec:propeller_uncovered}
The axial velocity outside the propeller slip stream $V_{xU}$ equals the apparent velocity (\autoref{eq:V_x_U_semiempirical}). A small contribution from the yaw rate is also added for twin screw ships (\autoref{eq:V_R_x_U_semiempirical}) so that the velocity outside the slip stream can be calculated with \autoref{eq:V_R_U_semiempirical}.
\begin{equation}
    \label{eq:V_x_U_semiempirical}
    V_{x U } = V_{A }
\end{equation}
%
\begin{equation}
    \label{eq:V_R_x_U_semiempirical}
    V_{R x U } = V_{x U } - r y_{R}
\end{equation}
%
\begin{equation}
    \label{eq:V_R_U_semiempirical}
    V_{R U } = \sqrt{V_{R x U }^{2} + V_{R y }^{2}}
\end{equation}
%
%
\subsubsection{Velocity in the propeller slip stream}
\label{sec:propeller_induced_velocity}
According to momentum theory the mean axial flow velocity far downstream of the propeller $V_{\infty}$ is given by \autoref{eq:V_infty_semiempirical} \cite{brix_manoeuvring_1993} where the thrust coefficient $C_{Th}$ is calculated with \autoref{eq:C_Th_semiempirical} where $r_0$ is the propeller radius and the apparent velocity $V_A$ is given by \autoref{eq:V_A_semiempirical}.
\begin{equation}
    \label{eq:V_infty_semiempirical}
    V_{\infty } = V_{A } \sqrt{C_{Th } + 1}
\end{equation}
%
\begin{equation}
    \label{eq:C_Th_semiempirical}
    C_{Th } = \frac{2 T_{}}{\pi V_{A }^{2} r_{0}^{2} \rho}
\end{equation}
%
\begin{equation}
    \label{eq:V_A_semiempirical}
    V_{A} = u \left(1 - w_{f}\right)
\end{equation}

The radius of the propeller slipstream far behind the propeller is given by \autoref{eq:r_infty_semiempirical}.
\begin{equation}
    \label{eq:r_infty_semiempirical}
    r_{\infty } = r_{0} \sqrt{\frac{V_{A }}{2 V_{\infty }} + \frac{1}{2}}
\end{equation}
The velocity and the radius of the propeller slipstream at the position of the rudder can be calculated with \autoref{eq:V_x_C_semiempirical} and \autoref{eq:r_p_semiempirical} where $x$ is the distance between the propeller and the rudder.
\begin{equation}
    \label{eq:V_x_C_semiempirical}
    V_{x C } = \frac{V_{\infty } r_{\infty }^{2}}{r_{x }^{2}}
\end{equation}
%
\begin{equation}
    \label{eq:r_p_semiempirical}
    r_{x } = \frac{r_{0} \left(\frac{r_{\infty } \left(\frac{x}{r_{0}}\right)^{1.5}}{r_{0}} + \frac{0.14 r_{\infty }^{3}}{r_{0}^{3}}\right)}{\left(\frac{x}{r_{0}}\right)^{1.5} + \frac{0.14 r_{\infty }^{3}}{r_{0}^{3}}}
\end{equation}
Turbulent mixing of the slipstream and the surrounding flow will increase the radius $r_x$ by $r_\Delta$(\autoref{eq:r_Delta_semiempirical}) so that a corrected axial velocity $V_{xcorr}$ can be calculated according to \autoref{eq:V_x_corr_semiempirical}.
\begin{equation}
    \label{eq:r_Delta_semiempirical}
    r_{\Delta } = \frac{0.15 x \left(- V_{A } + V_{x C }\right)}{V_{A } + V_{x C }}
\end{equation}
%
\begin{equation}
    \label{eq:V_x_corr_semiempirical}
    V_{x corr } = V_{A } + \frac{r_{x }^{2} \left(- V_{A } + V_{x C }\right)}{\left(r_{\Delta } + r_{x }\right)^{2}}
\end{equation}
For a twin screw ship a small contribution from the yaw rate is also added to the velocity as seen in \autoref{eq:V_R_x_C_semiempirical}.
\begin{equation}
    \label{eq:V_R_x_C_semiempirical}
    V_{R x C} = V_{x corr} - r y_{R}
\end{equation}
The velocity for the covered part of the rudder is obtained by \autoref{eq:V_R_C_semiempirical}.
\begin{equation}
    \label{eq:V_R_C_semiempirical}
    V_{R C} = \sqrt{V_{R x C}^{2} + V_{R y}^{2}}
\end{equation}
$V_{xcorr}$ is also used to calculate the lift diminished factor $\lambda_R$ together with the expressions in  \autoref{eq:lambda_R_semiempirical} to \autoref{eq:c_semiempirical}.
\begin{equation}
    \label{eq:lambda_R_semiempirical}
    \lambda_{R } = \left(\frac{V_{A }}{V_{x corr }}\right)^{f_{}}
\end{equation}
%
\begin{equation}
    \label{eq:f_semiempirical}
    f_} = \frac{512}{\left(2 + \frac{d_}}{c_}}\right)^{8}}
\end{equation}
%
\begin{equation}
    \label{eq:d_semiempirical}
    d_{} = \frac{\sqrt{\pi} \left(r_{\Delta } + r_{p }\right)}{2}
\end{equation}
\begin{equation}
    \label{eq:c_semiempirical}
    c_} = \frac{c_{r}}{2} + \frac{c_{t}}{2}
\end{equation}

\subsubsection{Rudder drag}
\label{sec:rudder_drag}
The total rudder drag $D_R$ is calculated as a sum of the contributions from the parts covered and uncovered by the propeller as seen in \autoref{eq:D_R_semiempirical}.
\begin{equation}
    \label{eq:D_R_semiempirical}
    D_{R } = 0.5 \rho \left(A_{R C} C_{D C } V_{R C }^{2} + A_{R U} C_{D U } V_{R U }^{2}\right)
\end{equation}
The drag coefficients for covered $C_{DC}$ and uncovered $C_{DU}$ are calculated with the similar equations: \autoref{eq:C_D_C_semiempirical}, and \autoref{eq:C_D_U_semiempirical}; where $C_{D0C}$ (\autoref{eq:C_D0_C_semiempirical}) and $C_{D0U}$ (\autoref{eq:C_D0_U_semiempirical}) are the drag at zero rudder angle  and $e_0 = 0.9$ is the Oswald efficiency factor. 
\begin{equation}
    \label{eq:C_D_C_semiempirical}
    C_{D C} = C_{D0 C} + \frac{C_{D tune} C_{L}^{2}}{\pi AR_{e} e_{0}}
\end{equation}
%
\begin{equation}
    \label{eq:C_D_U_semiempirical}
    C_{D U } = C_{D0 U } + \frac{C_{D tune} C_{L }^{2}}{\pi AR_{e } e_{0}}
\end{equation}
%
\begin{equation}
    \label{eq:C_D0_C_semiempirical}
    C_{D0 C} = 2.5 C_{D0 tune} C_{F C}
\end{equation}
%
\begin{equation}
    \label{eq:C_D0_U_semiempirical}
    C_{D0 U} = 2.5 C_{D0 tune} C_{F U}
\end{equation}
$C_{D0C}$ and $C_{D0U}$ will be different due to different Reynolds number $Re$ as seen in \autoref{eq:C_F_C_semiempirical} to \autoref{eq:Re_F_U_semiempirical}.
\begin{equation}
    \label{eq:C_F_C_semiempirical}
    C_{F C} = \frac{0.075 \log{\left(10 \right)}^{2}}{\log{\left(Re_{F C} - 2 \right)}^{2}}
\end{equation}
%
\begin{equation}
    \label{eq:C_F_U_semiempirical}
    C_{F U } = \frac{0.075 \log{\left(10 \right)}^{2}}{\log{\left(Re_{F U } - 2 \right)}^{2}}
\end{equation}
%
\begin{equation}
    \label{eq:Re_F_C_semiempirical}
    Re_{F C} = \frac{V_{R C} c_}}{\nu}
\end{equation}
%
\begin{equation}
    \label{eq:Re_F_U_semiempirical}
    Re_{F U} = \frac{V_{R U} c_}}{\nu}
\end{equation}
