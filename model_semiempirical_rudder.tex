The semi-empirical rudder model is a lifting line model. The surge and sway force are expressed as: rudder lift $L_R$, and rudder drag $D_R$; projected on the ship through the rudder inflow angle $\alpha_f$, as seen in \autoref{eq:X_R_semiempirical} and \autoref{eq:Y_R_semiempirical}.
This angle is a sum of the initial inflow to the rudder at a straight course $\gamma_0$ and the inflow to the rudder $\gamma$ due to: propeller induced speed, drift angle, and yaw rate of the ship; as seen in \autoref{eq:gamma_semiempirical}.
\begin{equation}
    \label{eq:X_R_semiempirical}
    X_{R } = - D_{R } \cos{\left(\alpha_{f } \right)} + L_{R } \sin{\left(\alpha_{f } \right)}
\end{equation}
%
\begin{equation}
    \label{eq:Y_R_semiempirical}
    Y_{R} = D_{R} \sin{\left(\alpha_{f} \right)} + L_{R} \cos{\left(\alpha_{f} \right)}
\end{equation}
%
\begin{equation}
    \label{eq:alpha_f_semiempirical}
    \alpha_{f } = \gamma_{0 } + \gamma_{}
\end{equation}
%
\begin{equation}
    \label{eq:gamma_semiempirical}
    \gamma_} = \operatorname{atan}{\left(\frac{V_{R y}}{V_{R x C}} \right)}
\end{equation}
The transverse velocity at the rudder $V_{Ry}$ is calculated with the ship's yaw rate $r$ and the transverse velocity $v$ multiplied by their flow straightening $\kappa_{rtot}$ and $\kappa_{vtot}$ (\autoref{eq:V_R_y_semiempirical}). The flow straightening has a nonlinear dependency of the geometric inflow angle $\gamma_g$ (\autoref{eq:gamma_g_semiempirical}) as seen in \autoref{eq:kappa_r_tot_semiempirical} and \autoref{eq:kappa_v_tot_semiempirical}, so that the flow straightening may vary for different inflow angles, which is an enhancement compared to the MMG model.
The axial velocity at the rudder $V_{RxC}$, including the velocity of the propeller race is presented in \autoref{sec:velocity_in_the_propeller_slip_stream}.
\begin{equation}
    \label{eq:V_R_y_semiempirical}
    V_{R y} = - \kappa_{r tot} r x_{R} - \kappa_{v tot} v
\end{equation}
%
\begin{equation}
    \label{eq:kappa_r_tot_semiempirical}
    \kappa_{r tot} = \kappa_{r} + \kappa_{r \gamma g} \left|{\gamma_{g}}\right|
\end{equation}
%
\begin{equation}
    \label{eq:kappa_v_tot_semiempirical}
    \kappa_{v tot} = \kappa_{v} + \kappa_{v \gamma g} \left|{\gamma_{g}}\right|
\end{equation}
%
\begin{equation}
    \label{eq:gamma_g_semiempirical}
    \gamma_{g} = \operatorname{atan}{\left(\frac{- r x_{R} - v}{V_{R x C}} \right)}
\end{equation}

The yaw moment is modelled as the sway force times a lever arm $x_R$ as seen in \autoref{eq:N_R_semiempirical}.
\begin{equation}
    \label{eq:N_R_semiempirical}
    N_{R} = Y_{R} x_{R}
\end{equation}
%
%
\subsubsection{Rudder lift}
\label{sec:rudder lift}
With the inspiration from the work of \citet{villa_numerical_2020}, the total rudder lift is calculated as the sum of the lift from the part of the rudder that is covered by the propeller race $L_{RC}$ and the uncovered part $L_{RU}$ as seen in \autoref{eq:L_R_semiempirical}.
\begin{equation}
    \label{eq:L_R_semiempirical}
    L_{R } = L_{R C } + L_{R U }
\end{equation}
The lift forces are calculated with the lift coefficient $C_L$ as seen in \autoref{eq:L_R_U_semiempirical} and \autoref{eq:L_R_C_semiempirical}. These equations are essentially the same except that the lift by the covered part $L_{RC}$ is diminished by the factor $\lambda_R$ (\autoref{eq:lambda_R_semiempirical}) due to the limited radius of the propeller slipstream in the lateral direction \citep{brix_manoeuvring_1993}. A tuning coefficient $C_{Ltune}$ has also been added to these equations.
\begin{equation}
    \label{eq:L_R_U_semiempirical}
    L_{R U } = \frac{A_{R U} C_{L } C_{L tune} V_{R U }^{2} \rho}{2}
\end{equation}
%
\begin{equation}
    \label{eq:L_R_C_semiempirical}
    L_{R C } = \frac{A_{R C} C_{L } C_{L tune} V_{R C }^{2} \lambda_{R } \rho}{2}
\end{equation}
The lift coefficient $C_L$ is calculated with semi-empirical formulas (see \autoref{sec:CL}). The velocities of the covered $V_{RC}$ and uncovered part $V_{RC}$ are calculated according to \autoref{sec:velocity_outside_the_propeller_slip_stream} and \autoref{sec:velocity_in_the_propeller_slip_stream}.
%
%
\subsubsection{Rudder drag}
\label{sec:rudder_drag}
The total rudder drag $D_R$ is calculated as a sum of the contributions from the parts covered and uncovered by the propeller as seen in \autoref{eq:D_R_semiempirical}.
\begin{equation}
    \label{eq:D_R_semiempirical}
    D_{R } = 0.5 \rho \left(A_{R C} C_{D C } V_{R C }^{2} + A_{R U} C_{D U } V_{R U }^{2}\right)
\end{equation}
The drag coefficients: $C_{DC}$, and $C_{DU}$ are calculated with semi-empirical formulas according to \autoref{sec:CD}.

