The semi-empirical rudder model is a lifting line model similar to \citet{kjellberg_sailing_2023, matusiak_dynamics_2021, hughes_tempest_2011} which is largely based on the rudder wind tunnel tests conducted by \citet{whicker_free-stream_1958}. The surge and sway force are expressed as: rudder lift $L_R$, and rudder drag $D_R$, which are projected on the ship through the rudder inflow angle $\alpha_f$ (see \autoref{eq:X_R_semiempirical}, \autoref{eq:Y_R_semiempirical}, and \autoref{fig:inflow}).
This angle is a sum of the initial inflow to the rudder at a straight course $\gamma_0$ and the inflow to the rudder $\gamma$ due to: propeller induced speed, drift angle, and yaw rate of the ship; as seen in \autoref{eq:gamma_semiempirical}.
%
\begin{figure}[h]
    \centering
    \includesvg[pretex=\centering\fontsize{10}{11}]{figures/rudder_flow.svg}
    \caption{Inflow to rudder.}
    \label{fig:inflow}
\end{figure}
%
\begin{equation}
    \label{eq:X_R_semiempirical}
    X_{R } = - D_{R } \cos{\left(\alpha_{f } \right)} + L_{R } \sin{\left(\alpha_{f } \right)}
\end{equation}
%
\begin{equation}
    \label{eq:Y_R_semiempirical}
    Y_{R} = D_{R} \sin{\left(\alpha_{f} \right)} + L_{R} \cos{\left(\alpha_{f} \right)}
\end{equation}
%
\begin{equation}
    \label{eq:alpha_f_semiempirical}
    \alpha_{f } = \gamma_{0 } + \gamma_{}
\end{equation}
%
\begin{equation}
    \label{eq:gamma_semiempirical}
    \gamma_} = \operatorname{atan}{\left(\frac{V_{R y}}{V_{R x C}} \right)}
\end{equation}
The transverse velocity at the rudder $V_{Ry}$ is calculated with the ship's yaw rate $r$ and the transverse velocity $v$ multiplied by their flow straightening $\kappa_{rtot}$ and $\kappa_{vtot}$ (\autoref{eq:V_R_y_semiempirical}). The flow straightening has a nonlinear dependency of the geometric inflow angle $\gamma_g$ (\autoref{eq:gamma_g_semiempirical}) as seen in \autoref{eq:kappa_r_tot_semiempirical} and \autoref{eq:kappa_v_tot_semiempirical}, so that the flow straightening may vary for different inflow angles, which is an enhancement compared to the MMG model.
The axial velocity at the rudder $V_{RxC}$, including the velocity of the propeller race, is presented in \autoref{sec:velocity_in_the_propeller_slip_stream}.
\begin{equation}
    \label{eq:V_R_y_semiempirical}
    V_{R y} = - \kappa_{r tot} r x_{R} - \kappa_{v tot} v
\end{equation}
%
\begin{equation}
    \label{eq:kappa_r_tot_semiempirical}
    \kappa_{r tot} = \kappa_{r} + \kappa_{r \gamma g} \left|{\gamma_{g}}\right|
\end{equation}
%
\begin{equation}
    \label{eq:kappa_v_tot_semiempirical}
    \kappa_{v tot} = \kappa_{v} + \kappa_{v \gamma g} \left|{\gamma_{g}}\right|
\end{equation}
%
\begin{equation}
    \label{eq:gamma_g_semiempirical}
    \gamma_{g} = \operatorname{atan}{\left(\frac{- r x_{R} - v}{V_{R x C}} \right)}
\end{equation}
The yawing moment is modelled as the sway force times a lever arm $x_R$ as seen in \autoref{eq:N_R_semiempirical}.
\begin{equation}
    \label{eq:N_R_semiempirical}
    N_{R} = Y_{R} x_{R}
\end{equation}
%
%
\subsubsection{Rudder lift}
\label{sec:rudder lift}
With inspiration from the work of \citet{villa_numerical_2020}, the total rudder lift is calculated as the sum of the lift from the part of the rudder that is covered by the propeller $L_{RC}$ and the uncovered part $L_{RU}$ as seen in \autoref{eq:L_R_semiempirical} and \autoref{fig:rudder_coverage}.
\begin{equation}
    \label{eq:L_R_semiempirical}
    L_{R } = L_{R C } + L_{R U }
\end{equation}
%
\begin{figure}[h]
    \centering
    \includesvg{figures/rudder_coverage.svg}
    \caption{Rudder areas covered and uncovered by the propeller.}
    \label{fig:rudder_coverage}
\end{figure}
%
The lift forces are calculated (\autoref{eq:L_R_U_semiempirical} and \autoref{eq:L_R_C_semiempirical}) with the lift coefficient $C_L$. These equations are essentially the same except that the lift by the covered part $L_{RC}$ is diminished by the factor $\lambda_R$ (\autoref{eq:lambda_R_semiempirical}) due to the limited radius of the propeller slipstream in the lateral direction \citep{brix_manoeuvring_1993}. A tuning coefficient $C_{Ltune}$ has also been added to these equations.
\begin{equation}
    \label{eq:L_R_U_semiempirical}
    L_{R U } = \frac{A_{R U} C_{L } C_{L tune} V_{R U }^{2} \rho}{2}
\end{equation}
%
\begin{equation}
    \label{eq:L_R_C_semiempirical}
    L_{R C } = \frac{A_{R C} C_{L } C_{L tune} V_{R C }^{2} \lambda_{R } \rho}{2}
\end{equation}
The velocities of the uncovered $V_{RU}$ and covered $V_{RC}$ parts of the rudder are calculated according to \ref{sec:velocity_outside_the_propeller_slip_stream}, and \ref{sec:velocity_in_the_propeller_slip_stream}.
For a none stalling rudder, the lift coefficient $C_L$ is calculated with \autoref{eq:C_L_semiempirical}.
\begin{equation}
    \label{eq:C_L_semiempirical}
    C_{L} = \lambda_{gap} \left(\alpha_} dC_{L dalpha} + \frac{C_{DC} \alpha_} \left|{\alpha_}}\right|}{AR_{e}}\right)
\end{equation}
%
\begin{equation}
    \label{eq:alpha_semiempirical}
    \alpha_{} = \\delta + \gamma_{0 } + \gamma_{}
\end{equation}
The effective aspect ratio $AR_e$ accounts for the mirror image effect when the rudder is flush with the hull, where the effective aspect ratio is typically assumed to be twice the geometric aspect ratio $AR_g$ (\autoref{eq:AR_e_semiempirical}) \citep{hughes_tempest_2011}.
The wPCC rudder is however not flush to the hull, so that a gap is created between the rudder and rudder horn for larger rudder angles. This gap reduces the pressure difference between the high and low pressure sides in the upper part of the rudder. \citet{matusiak_dynamics_2021} proposed that the gap effect can be modelled as a reduced aspect ratio. Instead a simpler approach is proposed in the this paper -- based on experience. A factor $\lambda_{gap}$ is introduced, which is calculated according to \autoref{eq:lambda_gap_semiempirical}. The gap effect is only activated above a threshold rudder angle $\delta_{lim}$, and the strength of the gap effect is modelled by a factor $s$ as seen in \autoref{fig:gap}.
\begin{equation}
    \label{eq:lambda_gap_semiempirical}
    C_{gap} = \begin{cases} 1 & \text{for}\: \delta_{lim} > \left|{\delta}\right| \\s \left(- \delta_{lim} + \left|{\delta}\right|\right)^{2} + 1 & \text{otherwise} \end{cases}
\end{equation}
\begin{figure}[h]
    \centering
    \includesvg[width=0.5\columnwidth]{figures/gap_effect.gap.svg}
    \caption{The rudder lift is reduced by the gap between rudder and rudder horn for larger rudder angles.}
    \label{fig:gap}
\end{figure}
The lift slope of the rudder $\frac{\partial C_L}{\partial \alpha}$ is calculated with \autoref{eq:dC_L_dalpha_semiempirical} where $a_0$ is the section lift curve slope (\autoref{eq:a_0_semiempirical}) and $\Omega$ is the sweep angle of the quarter chord line \citep{lewis_principles_1989}.
\begin{equation}
    \label{eq:dC_L_dalpha_semiempirical}
    dC_{L dalpha} = \frac{AR_{e} a_{0}}{\sqrt{\frac{AR_{e}^{2}}{\cos^{4}{\left(\Omega \right)}} + 4} \cos{\left(\Omega \right)} + 1.8}
\end{equation}
%
\begin{equation}
    \label{eq:AR_e_semiempirical}
    AR_{e} = 2 AR_{g}
\end{equation}
%
\begin{equation}
    \label{eq:AR_g_semiempirical}
    AR_{g} = \frac{b_{R}^{2}}{A_{R}}
\end{equation}
%
\begin{equation}
    \label{eq:a_0_semiempirical}
    a_{0 } = 1.8 \pi
\end{equation}
There is also a small nonlinear part to $C_L$ modelled by the cross flow drag coefficient $C_{DC}$ which is calculated for a rudder with squared tip using \autoref{eq:C_D_crossflow_semiempirical} where the taper ratio $\lambda$ is defined as the ratio between the chords at the tip and the root of the rudder (\autoref{eq:lambda__semiempirical}) \citep{hughes_tempest_2011}. 
\begin{equation}
    \label{eq:C_D_crossflow_semiempirical}
    C_{DC} = 1.6 \lambda^{} + 0.1
\end{equation}
%
\begin{equation}
    \label{eq:lambda__semiempirical}
    \lambda} = \frac{c_{t}}{c_{r}}
\end{equation}
%
%
\subsubsection{Rudder drag}
\label{sec:rudder_drag}
The total rudder drag $D_R$ is calculated as a sum of the contributions from the parts covered and uncovered by the propeller as seen in \autoref{eq:D_R_semiempirical}.
\begin{equation}
    \label{eq:D_R_semiempirical}
    D_{R } = 0.5 \rho \left(A_{R C} C_{D C } V_{R C }^{2} + A_{R U} C_{D U } V_{R U }^{2}\right)
\end{equation}
The drag coefficients: $C_{DC}$, and $C_{DU}$ are calculated with semi-empirical formulas according to \autoref{sec:CD}.

