\subsection{Prime system}
\label{sec:prime_system}
Variables within some of the equations in this paper are expressed using non dimensional units with the prime system and are then denoted by the prime symbol ($'$). Variables are converted from SI units to the prime system using the denominators in \autoref{tab:prime_system} for the corresponding physical quantity; where $U$, and $L$ are the velocity and length between perpendiculars of the ship, $\rho$ is the water density.
For the calculation of surge velocity $u'$ the perturbed velocity $u-U_0$ about a nominal speed $U_0$ is used instead of $u$ as seen in \autoref{eq:u_prime}. This is to avoid that $u'$ is 1 for all speeds when the ship is on a straight course (where $u=U$) as in a resistance or self propulsion test for instance. The usage of the perturbed velocity allows for higher order resistance terms in the model such as $X_{uu}$, which would otherwise not be possible. 
\begin{equation}
    \label{eq:u_prime}
    u' = \frac{u-U_0}{U}
\end{equation}
In order to have a non dimensional model, $U_0$ is instead expressed as a Froude number within the model (\autoref{eq:Fn0}), where $F_{n0}=0.02$ is used in this paper.
\begin{equation}
    \label{eq:Fn0}
    F_{n0} = \frac{U_0}{\sqrt{g \cdot L}}
\end{equation}

\begin{table}[h!]
    \centering
    \caption{Scalings with prime system}
    \label{tab:prime_system}
    \pgfplotstabletypeset[col sep=comma,
        columns={Physical quantity,SI unit,Denominator},
        columns/SI unit/.style={string type},
        columns/Physical quantity/.style={string type},
        columns/Denominator/.style={string type},
        column type=l,	% specify the align method
        every head row/.style={before row=\hline,after row=\hline},	% style the first row
        every last row/.style={after row=\hline},	% style the last row
    ]{tables/prime_system.prime_system.csv"}
\end{table}