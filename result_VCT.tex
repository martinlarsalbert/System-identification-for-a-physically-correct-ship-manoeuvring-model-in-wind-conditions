Predictions with the identified reference model were in good agreement with the corresponding VCT data, as shown in \autoref{fig:vct}. The ranges of variations were chosen to match the states of the model tests, where, for instance, the drift angle of 10 degrees is the largest recorded from the zigzag tests (\autoref{fig:vct_drift_angle}). The hull forces are almost linear for these small drift angles and yaw rates. In addition, the higher-order terms in the hull force model (\autoref{eq:X_H}--\autoref{eq:N_H}) were thus omitted in the VCT and ID regressions to reduce the multicollinearity.
\begin{figure}
    \centering
    \begin{subfigure}[b]{0.49\textwidth}
        \centering
        \includesvg[width=\textwidth]{figures/vct.VCT Circle.svg}
        \caption{Yaw rates.}
        \label{fig:vct_circle}
    \end{subfigure}
    \hfill
    \begin{subfigure}[b]{0.49\textwidth}
        \centering
        \includesvg[width=\textwidth]{figures/vct.VCT Drift angle.svg}
        \caption{Drift angles.}
        \label{fig:vct_drift_angle}
    \end{subfigure}
    %
    \begin{subfigure}[b]{0.49\textwidth}
        \centering
        \includesvg[width=\textwidth]{figures/vct.VCT Rudder angle.svg}
        \caption{Rudder angles.}
        \label{fig:vct_rudder_angle}
    \end{subfigure}
    \hfill
    \begin{subfigure}[b]{0.49\textwidth}
        \centering
        \includesvg[width=\textwidth]{figures/vct.VCT Thrust variation.svg}
        \caption{Thrust variations at 10 degrees rudder angle.}
        \label{fig:vct_thrust_variation}
    \end{subfigure}
    \caption{Comparison between the VCT data ($+$) and predictions with the reference model (lines) (expressed in the prime system $\times 1000$) for total damping force $D$ (red), hull force $H$ (green), and rudder forces $R$ (blue).}
    \label{fig:vct}
\end{figure}